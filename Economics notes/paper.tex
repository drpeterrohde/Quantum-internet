\documentclass[aps,pra,twocolumn,amsmath,amssymb,nofootinbib,superscriptaddress]{revtex4}

\newcommand{\bra}[1]{\langle#1|}
\newcommand{\ket}[1]{|#1\rangle}
\newcommand{\op}[2]{\hat{\textbf{#1}}_{#2}}
\newcommand{\dagop}[2]{\hat{\textbf{#1}}_{#2}^\dag}
\usepackage[pdftex]{graphicx}
\usepackage{mathrsfs}
\usepackage[colorlinks]{hyperref}

\begin{document}

\bibliographystyle{apsrev}

%
% Title
%

\title{Notes on economics}

%
% Authors
%

\date{\today}

\frenchspacing

%
% Abstract
%

\begin{abstract}
\end{abstract}

\maketitle 

\section{Forward contract pricing model}

We wish to price the future value of a fixed number of qubits, that are traded as a time-shared asset, unified with the global network.

The standard forward pricing model is,
\begin{align}
F(T) = S_0 e^{(r-q)T} - \sum_{t=0}^T D_t e^{(r-q)(T-t)},
\end{align}
where $T$ is the time of maturity (at which the contract is executed), $r$ is the risk-free rate of return, $q$ is the cost of carry, and $D_t$ is the expected dividend at time $t$.

Assume cost-of-carry, $q$, is zero since maintenance of physical computer assets is negligible and the goods are effectively non-perishable.

Assume that the dollar value of FLOPS, $L_t$, is decreasing exponentially over time $t$. This is what we observe for the classical Moore's Law, and it is reasonable to assume a similar exponential trajectory in the future,
\begin{align}
	L_t = L_0{\gamma_L}^{-t},
\end{align}
where \mbox{$\gamma_L\geq 1$} characterises the rate of exponential decay.

Let the number of FLOPS associated with the tradable asset grow according to the quantum economic leverage over time, $\lambda_t$. That is, as the future network grows, so does our leverage, and thus our classical-equivalent processing power.

Assume the number of qubits in the global network in the future is growing exponentially over time (i.e the rate of progress of quantum technology will observe a Moore's Law-like behaviour -- exponential reduction in qubit manufacturing costs over time will yield exponential growth in the number of qubits in existence),
\begin{align}
	N_t = N_0 {\gamma_N}^{t},
\end{align}
where \mbox{$\gamma_N\geq 1$} characterises the rate of exponential growth in the number of qubits available to the quantum network.

The leverage then scales as,
\begin{align}
\lambda_t &= \eta_n \frac{f_\text{sc}(N_t)}{N_t} \nonumber \\
&= \eta_n \frac{f_\text{sc}(N_0 {\gamma_N}^{t})}{N_0 {\gamma_N}^{t}},
\end{align}
where,
\begin{align}
	\eta_n = \frac{n}{f_\text{sc}(n)},
\end{align}
and $n$ is the number of qubits involved in the transaction, which we treat as a constant, since we are valuing the future price of an asset comprising a fixed number of qubits.

Let the dividend, $D_t$, be a measure of the dollar value of the $n$ qubits' computational power at a given time $t$. This scales with the leverage and dollars-per-FLOPS,
\begin{align}
D_t &= L_t \lambda_t \nonumber \\
&= \frac{\eta_n L_0 {\gamma_L}^{-t} f_\text{sc}(N_t)}{N_t} \nonumber \\
&= \frac{nL_0}{f_\text{sc}(n)} \cdot \frac{f_\text{sc}(N_0 {\gamma_N}^{t})}{N_0(\gamma_L\gamma_N)^{t}}.
\end{align}

Let the spot price, $S_0$, be the present day price ($t=0$) of FLOPS times the leverage at time $T$ (i.e its future computing power),
\begin{align}
S_0 &= L_0 \lambda_T \nonumber \\
&= L_0 \eta_n \frac{f_\text{sc}(N_T)}{N_T} \nonumber \\
&= L_0 \eta_n \frac{f_\text{sc}(N_0 {\gamma_N}^{T})}{N_0 {\gamma_N}^{T}}
\end{align}

Then the forward price is,
\begin{align}
F(T) = \frac{e^{rT}L_0\eta_n}{N_0} \left[ \frac{f_\text{sc}(N_0 {\gamma_N}^{T})}{{\gamma_N}^{T}} - \sum_{t=0}^{T} \frac{f_\text{sc}(N_0 {\gamma_N}^{t})}{(\gamma_N \gamma_L e)^{t}} \right].
\end{align}

\subsection{Linear scaling functions}

As a first example, to provide a benchmark, let the scaling function be linear in $n$,
\begin{align}
f_\text{sc}(n) = \alpha n,	
\end{align}
as we would (approximately) observe for clustered classical computers, whereby there is no leverage. Then the forward price reduces to,
\begin{align}
F(T) &= {e^{rT}L_0} \left[1 - \sum_{t=0}^{T} (\gamma_L e)^{-t} \right] \nonumber \\
&= {e^{rT}L_0} \left[\frac{1-(e\gamma_L)^{-T}}{1-e\gamma_L} \right].
\end{align}

\subsection{Quadratic scaling functions}

Let the scaling function be a simple quadratic polynomial (e.g the quantum resources are being employed for Grover-like algorithms, such as speeding up \textbf{NP}-complete optimisation problems),
\begin{align}
f_\text{sc}(n) = \alpha n^2	
\end{align}
Then the forward price reduces to,
\begin{align}
F(T) &= \frac{e^{rT}L_0N_0}{n^2} \left[{\gamma_N}^{T} - \sum_{t=0}^{T} \left(\frac{\gamma_N}{\gamma_L e}\right)^t \right] \nonumber \\
&= \frac{e^{rT}L_0 N_0}{n^2} \left[{\gamma_N}^T + \frac{\gamma_N\left(\frac{\gamma_N}{e \gamma_L}\right)^T-e\gamma_L}{e\gamma_L - \gamma_N} \right].
\end{align}

\textbf{Something I don't understand: arbitrage should enforce the condition that 2 qubits costs twice as much as 1 qubit. This formula doesn't reflect this.}

\subsection{Exponential scaling functions}

\textbf{To do}

\end{document}
