\documentclass[aps,pra,twocolumn,amsmath,amssymb,nofootinbib,superscriptaddress]{revtex4}

\newcommand{\bra}[1]{\langle#1|}
\newcommand{\ket}[1]{|#1\rangle}
\newcommand{\op}[2]{\hat{\textbf{#1}}_{#2}}
\newcommand{\dagop}[2]{\hat{\textbf{#1}}_{#2}^\dag}
\usepackage[pdftex]{graphicx}
\usepackage{mathrsfs}
\usepackage[colorlinks]{hyperref}

\begin{document}

\bibliographystyle{apsrev}

%
% Title
%

\title{Notes on economics}

%
% Authors
%

\date{\today}

\frenchspacing

%
% Abstract
%

\begin{abstract}
\end{abstract}

\maketitle 

%
% Forward Contract Pricing Model
%

\section{Forward contract pricing model}\index{Forward contract pricing model}

Unlike conventional securities traded on the financial markets, the dividends paid by a computer are measured in units of computation, FLOPS, rather than dollars. However, of course there will always be some market for trading FLOPS for dollars. We envisage a derivatives market where the currency is in terms of present and future units of computation, expressed in its dollar-equivalent. Specifically, units of computation at different points in time are traded as a financial instrument.

Forward contracts such as these are immensely useful in conventional markets, as a means by which to secure future use or ownership of an asset at predictable points in time. For example, farmers make heavy use of forward contracts to lock in sale of their produce before it has been harvested, such that the value is locked in in advance and the sale is guaranteed. We envisage similar utility in the context of quantum computing. A company engaging in heavy use of computing power might have a need to perform certain computations on a regular basis at predictable intervals going forward. In this instance, forward contracts could be very helpful in reducing exposure to risk and guaranteeing access to the technology when needed.

Now let us price the future value of a fixed number of qubits, that are traded as a time-shared asset, unified with the global network. 

The standard forward pricing model is,
\begin{align} \index{Forward contract pricing model}
F(T) = S_0 e^{(r-q)T} - \sum_{t=0}^T D_t e^{(r-q)(T-t)},
\end{align}
where,
\begin{itemize}
	\item $F(T)$ is the forward price at time $T$.\index{Forward price}
	\item $T$ is the time of maturity, at which time the contract is executed and the transaction is made.\index{Time of maturity}
	\item $r$ is the risk-free rate of return.\index{Risk-free rate of return}
	\item $q$ is the cost of carry (e.g storage and maintenance costs). We assume this is zero since maintenance of physical computer assets is negligible and the goods are effectively non-perishable.\index{Cost of carry}
	\item $D_t$ is the expected dividend at time $t$.\index{Dividend}
\end{itemize}

The execution of computations typically has monetary value to the consumer. After all, they are paying hard-earned money for access to the technology. We let $L_t$ be the dollar value of FLOPS at time $t$. The exact form of this will be highly market-dependent. However, as a toy model we assume $L_t$ is decreasing exponentially over time. This is what we observe for the classical Moore's Law\index{Moore's Law}, and it is reasonable to assume a similar exponential trajectory in future technologies. Thus,
\begin{align}\index{Cost of computation}
	L_t = L_0{\gamma_L}^{-t},
\end{align}
where \mbox{$\gamma_L\geq 1$} characterises the rate of exponential decay.

Assume the number of qubits in the global network in the future is growing exponentially over time, i.e the rate of progress of quantum technology will observe a Moore's Law-like behaviour -- exponential reduction in qubit manufacturing costs over time will yield exponential growth in the number of qubits in existence. This is a reasonable assumption based on the observation of this ubiquitous kind of behaviour in present-day technologies. Classical computing has been on an exponential trajectory since the 1980's, and although it must eventually asymptote, it shows no sign of doing so in the immediate future. Thus, we let the number of qubits in the network be,
\begin{align}
	N_t = N_0 {\gamma_N}^{t},
\end{align}
where \mbox{$\gamma_N\geq 1$} characterises the rate of exponential growth in the number of qubits available to the quantum network.

The time-dependent quantum economic leverage scales as,
\begin{align}\index{Quantum economic leverage}
\lambda_n(t) = \frac{f_\text{sc}(N_0 {\gamma_N}^{t})\cdot n}{f_\text{sc}(n) \cdot N_0 {\gamma_N}^{t}},
\end{align}
where $n$ is the number of qubits involved in the transaction, which we treat as a constant, since we are valuing the future price of an asset comprising a fixed number of qubits.

Let the spot price, $S_0$, be the present-day price ($t=0$) of FLOPS times the share of computing power at time of maturity, $T$ (i.e future computing power, accounting for growth, priced according to present market value),
\begin{align}
S_0 &= L_0 \cdot c_n(T) \nonumber \\
&= L_0 r_n(T) \cdot f_\text{sc}(N_0 {\gamma_N}^{T}) \nonumber \\
&= \frac{L_0 n\cdot f_\text{sc}(N_0 {\gamma_N}^{T})}{N_0 {\gamma_N}^{T}}.
\end{align}

The computational dividend at time $t$, $D_t$, is masterfully baked using the same recipe,
\begin{align}
D_t &= \frac{L_0 {\gamma_L}^{-t} n\cdot f_\text{sc}(N_0 {\gamma_N}^{t})}{N_0 {\gamma_N}^{t}}.
\end{align}

Then the forward price for time-shared quantum computing is then,

DEFINITION XXXX

\begin{align}
F(T) &= S_0 e^{rT} - \sum_{t=0}^T D_t e^{r(T-t)} \nonumber \\
&= \frac{L_0 n e^{rT}}{N_0}\left[ \frac{f_\text{sc}(N_0 {\gamma_N}^{T})}{{\gamma_N}^T} - \sum_{t=0}^T \frac{f_\text{sc}(N_0 {\gamma_N}^{t})}{e^{rt}{\gamma_N}^t{\gamma_L}^{t}}\right].
\end{align}

\subsection{Example pricing dynamics}

Let us consider the behaviour of forward contract pricing dynamics for several different scaling functions of particular interest in quantum computing.

We will choose the variables as follows for simplicity: \mbox{$L_0=N_0=n=r=1$}, \mbox{$\gamma_N=\gamma_L=2$}. In this example scenario the forward price reduces to,
\begin{align}
F(T) = e^{T}\left[ \frac{f_\text{sc}(2^{T})}{2^T} - \sum_{t=0}^T \frac{f_\text{sc}(2^{t})}{(4e)^t}\right]
\end{align}

%
% Linear Scaling Functions
%

\subsection{Linear scaling functions}\index{Linear scaling functions}

To provide a benchmark against classical computation, let the scaling function be linear in $n$,
\begin{align}
f_\text{sc}(n) = n,	
\end{align}
Then the forward price reduces to,
\begin{align}
F(T) = e^{T}\left[ 1 - \sum_{t=0}^T \frac{1}{(2e)^t}\right].
\end{align}

%
% Quadratic Scaling Functions
%

\subsection{Quadratic scaling functions}\index{Quadratic scaling functions}

Let the scaling function be a simple quadratic polynomial (e.g the quantum resources are being employed for Grover-like algorithms, such as speeding up \textbf{NP}-complete optimisation problems). The scaling function is,
\begin{align}
f_\text{sc}(n) = n^2,	
\end{align}
Now the forward contract pricing model reduces to,
\begin{align}
F(T) = e^{T}\left[ 2^{T} - \sum_{t=0}^T \frac{1}{e^t}\right]
\end{align}

%
% Exponential Scaling Functions
%

\subsection{Exponential scaling functions}\index{Exponential scaling functions}

Finally let us consider exponential scaling functions -- the most powerful we might hope for. We let,
\begin{align}
f_\text{sc}(n) = e^n.	
\end{align}
And the forward price reduces to,
\begin{align}
F(T) = e^{T}\left[ \frac{\text{exp}(2^{T})}{2^T} - \sum_{t=0}^T \frac{\text{exp}(2^{t})}{(4e)^t}\right]
\end{align}

\end{document}
