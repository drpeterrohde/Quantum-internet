\documentclass[aps,pra,twocolumn,amsmath,amssymb,nofootinbib,superscriptaddress]{revtex4}

\usepackage[pdftex]{graphicx}
\usepackage{mathrsfs}
\usepackage[colorlinks]{hyperref}
\usepackage{braket}

\begin{document}

\part{Introduction to quantum mechanics}

%
% Overview
%

\section{Overview}

%
% Quantum States
%

\section{Quantum states}

%
% State Vectors
%

\subsection{State vectors}

\begin{align}
	\ket\psi &= \left(\begin{matrix}{}
	\alpha_1\\
	\alpha_2\\
	\vdots
\end{matrix}\right)\nonumber\\
	&= \sum_n \alpha_n \ket{n},
\end{align}
where \mbox{$\alpha_n\in\mathbb{C}$}, and for normalisation,
\begin{align}\label{eq:state_norm_cond}
\sum_n |\alpha_n|^2 = 1,	
\end{align}

\begin{align}
\ket\psi &= \left(\begin{matrix}{}
	\alpha \\
	\beta
\end{matrix}\right)\nonumber\\
&= \alpha\ket{0} + \beta\ket{1}.
\end{align}

\begin{align}
\bra\psi &= \left(\begin{matrix}{}
	\alpha \\
	\beta
\end{matrix}\right)^\dag\nonumber\\
&=\left(\begin{matrix}{}
	\alpha^*, \beta^*
\end{matrix}\right)\nonumber\\
&= \alpha^*\bra{0} + \beta^*\bra{1}.
\end{align}

\begin{align}
	\ket\psi = \sum_n \alpha_n \ket{n},\nonumber\\
	\ket\phi = \sum_n \beta_n \ket{n},
\end{align}

The \textit{overlap} between two states is defined as,
\begin{align}
\braket{\psi|\phi} = \sum_n \alpha_n\beta_n^*,
\end{align}
The normalisation condition from Eq.~(\ref{eq:state_norm_cond}) implies,
\begin{align}
\braket{\psi|\psi} = 1.	
\end{align}

Because basis states are orthonormal, this implies that for basis states $\ket{m}$ and $\ket{n}$,
\begin{align}
	\braket{m|n}=\delta_{m,n}.
\end{align}

%
% Composite Systems
%

\subsection{Composite systems}

\begin{align}
\ket\psi_{A,B} &= \ket\psi_A \otimes \ket\phi_B\nonumber\\
&=\sum_{m,n} \alpha_m\beta_n\ket{m}\otimes\ket{n}.
\end{align}

\begin{align}
\ket\psi_{A,B} =\sum_{m,n} \lambda_{m,n}\ket{m}\otimes\ket{n}.
\end{align}
In general, $\lambda_{m,n}$ may not be separable as \mbox{$\lambda_{m,n}=\alpha_m\beta_n$}.
\begin{align}
\ket\psi_{A,B} &= \frac{1}{\sqrt{2}}\left(\begin{matrix}{}
  1\\
  0\\
  0\\
  1
\end{matrix}\right)\nonumber\\
&= \frac{1}{\sqrt{2}}(\ket{0}_A\ket{0}_B+\ket{1}_A\ket{1}_B).
\end{align}
Cannot be written in separable form as \mbox{$\ket\psi_{A,B} = \ket\psi_A \otimes \ket\phi_B$}. This is a so-called \textit{entangled state}, whereby the two subsystems exhibit a type of quantum correlation with no classical analogue.

%
% Density Operators
%

\subsection{Density operators}

For an $n$-dimensional Hilbert space, the density operator, $\hat\rho$, is an \mbox{$n\times n$} complex Hermitian matrix, satisfying,
\begin{align}
	\hat\rho = \hat\rho^\dag.
\end{align}

\begin{align}
\hat\rho = \left(\begin{matrix}{}
  a & c \nonumber\\
  c^* & b
\end{matrix}\right).
\end{align}

\begin{align}
\mathrm{tr}(\hat\rho)=\sum_i \hat\rho_{i,i} = 1,	
\end{align}
for normalisation.

\begin{align}
\hat\rho &= \ket\psi\bra\psi\nonumber\\
&= \left(\begin{matrix}{}
  |\alpha|^2 & \alpha\beta \nonumber\\
  \alpha^*\beta^* & |\beta|^2
\end{matrix}\right).
\end{align}

\begin{align}
	\hat\rho = \sum_i p_i \hat\rho_i,
\end{align}
where the probabilities are normalised such that,
\begin{align}
	\sum_i p_i = 1.
\end{align}

Purity,
\begin{align}
\mathcal{P} = \mathrm{tr}(\hat\rho^2),
\end{align}
where \mbox{$\mathcal{P}=1$} only for pure states.

%
% Reduced States
%

\subsection{Reduced states}

\begin{align}
\hat\rho_A = \mathrm{tr}_B(\hat\rho_{A,B}).	
\end{align}

\begin{align}
\mathrm{tr}_B(\hat\rho_A\otimes\hat\rho_B) = \mathrm{tr}(\hat\rho_B) \cdot \hat\rho_A.
\end{align}

From the cyclic property of the trace, it follows that,
\begin{align}
\mathrm{tr}(\ket\psi\bra\phi) = \braket{\psi|\phi}.	
\end{align}

%
% Evolution
%

\section{Evolution}

\begin{align}
\hat{U} = e^{-i\hat{H}t}.	
\end{align}

%
% Measurement
%

\section{Measurement}

Measurement operator $\hat{M}$, let the eigenvectors be the so-called \textit{measurement projectors},
\begin{align}
\hat{M}_i = \ket{m_i}\bra{m_i},	
\end{align}
where,
\begin{align}
\sum_i \hat{M}_i = \hat{\mathbb{I}}.	
\end{align}

\begin{align}
p_i = \mathrm{tr}(\hat{M}_i\hat\rho).
\end{align}

\begin{align}
\hat\rho_i = \frac{\hat{M}_i\hat\rho\hat{M}_i^\dag}{\mathrm{tr}(\hat{M}_i\hat\rho\hat{M}_i^\dag)}.
\end{align}


\end{document}
