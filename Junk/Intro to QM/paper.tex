\documentclass[aps,pra,twocolumn,amsmath,amssymb,nofootinbib,superscriptaddress]{revtex4}

\usepackage[pdftex]{graphicx}
\usepackage{mathrsfs}
\usepackage[colorlinks]{hyperref}
\usepackage{braket}

\begin{document}

\part{Introduction to quantum mechanics}

%
% Overview
%

\section{Overview}

%
% Quantum States
%

\section{Quantum states}

%
% State Vectors
%

\subsection{State vectors}

\begin{align}
\ket\psi &= \left(\begin{matrix}{}
	\alpha \\
	\beta
\end{matrix}\right)\nonumber\\
&= \alpha\ket{0} + \beta\ket{1}.
\end{align}

\begin{align}
\bra\psi &= \left(\begin{matrix}{}
	\alpha \\
	\beta
\end{matrix}\right)^\dag\nonumber\\
&=\left(\begin{matrix}{}
	\alpha^*, \beta^*
\end{matrix}\right)\nonumber\\
&= \alpha^*\bra{0} + \beta^*\bra{1}.
\end{align}

\begin{align}
	\ket\psi = \sum_n \alpha_n \ket{n},\nonumber\\
	\ket\phi = \sum_n \beta_n \ket{n},
\end{align}

The \textit{overlap} between two states is defined as,
\begin{align}
\braket{\psi|\phi} = \sum_n \alpha\beta^*,
\end{align}
where,
\begin{align}
\sum_n |\alpha_n|^2 = 1,	
\end{align}
for normalisation. This is equivalent to writing,
\begin{align}
\braket{\psi|\psi} = 1.	
\end{align}

%
% Density Operators
%

\subsection{Density operators}

\begin{align}
\hat\rho = \left(\begin{matrix}{}
  a & c \nonumber\\
  c^* & b
\end{matrix}\right).
\end{align}

\begin{align}
\mathrm{tr}(\hat\rho)=1,	
\end{align}
for normalisation.

\begin{align}
\hat\rho &= \ket\psi\bra\psi\nonumber\\
&= \left(\begin{matrix}{}
  |\alpha|^2 & \alpha\beta \nonumber\\
  \alpha^*\beta^* & |\beta|^2
\end{matrix}\right).
\end{align}

\begin{align}
	\hat\rho = \sum_i p_i \hat\rho_i,
\end{align}
where the probabilities are normalised such that,
\begin{align}
	\sum_i p_i = 1.
\end{align}

%
% Reduced States
%

\subsection{Reduced states}

\begin{align}
\hat\rho_A = \mathrm{tr}_B(\hat\rho_{A,B}).	
\end{align}

\begin{align}
\mathrm{tr}_B(\hat\rho_A\otimes\hat\rho_B) = \mathrm{tr}(\hat\rho_B) \cdot \hat\rho_A.
\end{align}

From the cyclic property of the trace, it follows that,
\begin{align}
\mathrm{tr}(\ket\psi\bra\phi) = \braket{\psi|\phi}.	
\end{align}

%
% Evolution
%

\section{Evolution}

%
% Measurement
%

\section{Measurement}

\end{document}
