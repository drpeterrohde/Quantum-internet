\begin{table*}[!htb]
\caption{} \label{tab:}
\begin{tabular}{|p{0.755\linewidth}|p{0.22\linewidth}|}
	\hline
	Summary & References \& years \\
	\hline \hline
	a & b \\
\end{tabular}
\end{table*}

---

QKD

%The main drawback is thus the ability to perform long-distance transmission of photons. Current QKD networks have been limited to relatively small distances, on the order of $\sim100$km, including in Austria, Switzerland, Japan, USA, and China \cite{bib:lo2014secure}. The longest distance QKD network that is currently planned is the Beijing to Shanghai QKD link, spanning a distance of $\sim2,000$km. This involves 32 trusted nodes\index{Trusted nodes} to break the full route into shorter segments.

%Utilising space communications for the purpose of QKD has been discussed in several works \cite{bib:hughes2000quantum, bib:rarity2002ground, bib:pfennigbauer2003free, bib:aspelmeyer2003long, bib:armengol08}. As already demonstrated in space-based entanglement experiments \cite{bib:yin2017satellite, bib:ren2017ground, bib:liao2017satellite} far lower attenuation rates are possible than ground-only approaches. Since cryptography schemes such as BB84\index{BB84 protocol} do not require entanglement, these would appear to be the first widespread commercial application for quantum technologies.  

%The ready realisability of space based QKD was already noted in a variety of configurations including ground-to-space\index{Ground-to-space communication} and space-to-ground\index{Space-to-ground communication} quantum communication \cite{bib:rarity2002ground, bib:aspelmeyer2003long}. In the context of security, the first long-distance experiments that are likely to be demonstrated will employ trusted nodes\index{Trusted nodes}. For example, after performing QKD between satellite\index{Satellites} and ground stations\index{Ground stations}, a satellite could store the key for some time in a quantum memory\index{Quantum memory} until another QKD can be performed to another ground station using a one-time pad\index{One-time pad cipher} \cite{bib:liao2017satellite}. These types of experiments are currently planned to eventually facilitate intercontinental QKD\index{Intercontinental QKD} between China and Austria.  

%QKD has been widely experimentally demonstrated over long distances \cite{bib:Muller96}, and unlike quantum computing, QKD is at the stage of commercial viability. Thus, a quantum internet with low cost metrics would already find substantial utility with today's technology. Currently, great progress in being made in the implementation of QKD in fibre \cite{???}, over free-space \cite{bib:Buttler00}, and even over intercontinental satellite uplinks \cite{JWP}. It seems extremely likely that some government agencies would be rolling out QKD systems \cite{bib:Secret}, especially in light of the paranoia surrounding quantum codebreaking.