\documentclass[pra,onecolumn,preprintnumbers,superscriptaddress]{revtex4} 

\usepackage{graphicx}
\usepackage{bm}
\usepackage{amsmath,hyperref}
\usepackage{amssymb}
\usepackage{amsfonts}
\usepackage{fancyhdr}
\usepackage{slashed}
\usepackage{latexsym,epsfig,bbm}

\renewcommand{\baselinestretch}{1} 
\newcommand{\md}{\mathrm{d}}
\newcommand{\bra}[1]{\langle{#1}|}
\newcommand{\ket}[1]{|{#1}\rangle}
\newcommand{\brkt}[2]{\langle{#1}|{#2}\rangle}
\newcommand{\braket}[2]{\langle{#1}|{#2}\rangle}
\newcommand{\bopk}[3]{\langle{#1}|{#2}|{#3}\rangle}
\newcommand{\leftexp}[2]{{\vphantom{#2}}^{#1}{#2}}
\newcommand{\fv}[1]{\textsf{#1}}
\newcommand{\pd}[2]{\frac{\partial #1}{\partial #2}}
\newcommand{\td}[2]{\frac{\md #1}{\md #2}}
\newcommand{\utilde}[1]{\underset{\widetilde{}}{#1}}
\newcommand{\sv}[1]{\utilde{\bm{#1}}}
\newcommand{\del}{\nabla}
\newcommand{\boxdel}{\square}
\newcommand{\realsum}{\displaystyle\sum}
\newcommand{\tens}[1]{\mathbb{#1}}
\newcommand{\figref}[1]{Fig.~\ref{#1}}
\newcommand{\figsref}[1]{Figs.~\ref{#1}}
\def \d {\mathrm{d}}
\newcommand{\Tr}{\mathrm{Tr}}
\usepackage{color}
\definecolor{blue}{rgb}{0,0.2,1}
\newcommand{\blue}[1]{\textcolor{blue}{{{#1}}}}
\definecolor{red}{rgb}{0.9,0,0}
\newcommand{\red}[1]{\textcolor{red}{{{#1}}}}



\begin{document}

\title{The quantum internet: Verification of quantum computation}

\date{\today}

\begin{abstract}

\end{abstract}
\maketitle 
\section{Introduction}
\subsection{Relevance of verification of quantum computation for the quantum internet} 
Adversarial settings for non-NP problems...etc \\

Quantum `software' 

\subsection{Definition of verification} 
Mention the algorithms this is relevant for (e.g., \textit{not} NP) 
\subsection{Relationship to other kinds of verification}
Hypothesis testing; self-analysis; randomised bench-marking; state certification; authentication (ask Si-Hui) 
\section{Verification of universal quantum computation} 
Also mention relationship to blind quantum computation: only 2 examples of verifiable computing schemes that are not naturally blind 
\subsection{Two-party verification}
\subsubsection{MBQC and traps}
\subsubsection{Measurement-only verification}
Also mention relationship to state certification 
\subsubsection{Multi-party verification}
\section{Verification of non-universal models}
List non-universal models; 
\subsection{Verification of quantum simulation}
\subsection{DQC1}
\subsection{Boson sampling}
Circumstantial tests; 
\subsection{IQP}
Michael Bremner, Jozsa, Shepherd: sampling problem; Also Bremner (lattice model?) 
\subsection{Others}
\section{Further work}
\subsection{Continuous variables}
Example: advantage of measurement-only scheme: can be extended to an arbitrary size network and security of any one party is not compromised. Note also that a quantum software program is a particular quantum state that enables a quantum computer to perform a specific task (Preskill). So might think of the cubic states as a kind of quantum program? Consider the scenario that every downloaded state costs something. 
\subsection{Verification and quantum machine learning algorithms}
\subsection{Security in distributed computing}




%\bibliography{qmlveriref}

\end{document}
