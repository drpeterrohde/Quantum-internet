\def\pubmode{1}

\if 1\pubmode
	\documentclass[aps,pra,singlecolumn,superscriptaddress]{revtex4-1}
\fi

\if 2\pubmode
	\documentclass[aps,pra,twocolumn,superscriptaddress]{revtex4-1}
\fi


%\documentclass[aps,prl,reprint,groupedaddress]{revtex4-1}
\usepackage{graphicx}% Include figure files
\usepackage{dcolumn}% Align table columns on decimal point
\usepackage{bm}% bold math
\usepackage{amsmath}
\usepackage{indentfirst}
\usepackage{float}
\usepackage[colorlinks]{hyperref}
\bibliographystyle{apsrev4-1}

\newcommand{\bra}[1]{\langle#1|}
\newcommand{\ket}[1]{|#1\rangle}

\begin{document}
\maketitle

\section{Quantum search}

The problem of finding specific entries in unstructured data spaces is a ubiquitous one. This class of \textit{search algorithms} have amongst the broadest applicability of any class of algorithms. Computer scientists have invested excruciating man-hours\index{SJW} into organising and structuring data so as to minimise the resource overhead (in time or space) associated with extracting desired components. However, the methodology for achieving this, and the favourability of associated resource overheads, is highly dependent on the structure of the underlying data, or whether there even is any. To this end, numerous data structures and algorithms have been developed, accommodating for every mutation and variation of the problem imaginable. Often, there is a tradeoff between the overheads induced in time and memory.

For example, \textit{hash tables}\index{Hash tables} enable theoretical $O(1)$ lookup times on data with a \textit{key-value pair}\index{Key-value pair} data structure. In a key-value pair each data entry (value) is tagged with a unique identifier (key) used for lookup. For example, when storing telephone numbers one might represent entries as key-value pairs, where the keys are people's names, and the values their respective telephone numbers. An efficient algorithm for mapping keys to physical memory addresses would imply efficient lookup of telephone numbers by name.

In the absence of a key-value representation one might simply store data in sorted form. However, this requires pre-sorting the entire data space, which may become costly for large data sets, and requires continual rearrangement whenever the data space is modified, making it computationally costly for mutable data.

For the end user, who wishes to find data elements, the worst-case data space is one with no order or underlying structure. In this instance, it can easily be seen that the best one can hope for, in terms of algorithmic runtime, is to simply look through the data space brute-force\index{Brute-force} until we find what we are looking for. For example, using the earlier analogy of the telephone book, if you were to try and use my phone number to find my name, the best you could do is start at page one and keep going until you found my name. It is clear that with an unstructured space of $N$ elements, this brute-force search algorithm requires on average $O(N)$ queries to find the desired entry. We call this the \textit{unstructured search problem}.

The brute-force classical algorithm, despite already being technically `efficient' (i.e $O(N)$ runtime), could nonetheless become unwieldy for very large datasets. The quantum search algorithm presents as solution to this problem using only $O(\sqrt{N})$ runtime, a quadratic enhancement. Whilst this falls far short of the exponential quantum enhancement one might have hoped for, it is nonetheless still extremely helpful for many purposes, given broad applications for solving this problem.

We will formulate the quantum search algorithm as an oracular algorithm, where the oracle takes as input an $n$-bit string, and outputs 1 if the input matches the entry we are looking for, otherwise 0. This formulation of the problem makes the algorithm naturally suited to solving satisfiability problems (many of which are \textbf{NP}-complete).





%%%%

Classically, searching an unsorted database of $N$ items requires $O(N)$ time. However, performing the search using the quantum algorithm of Grover's algorithm takes only $O(\sqrt{N})$ time \cite{grover1997quantum}, which is optimal for searching an unsorted database \cite{bennett1997strengths, zalka1999grover}. The ``time" here is defined as the number of queries to the oracle. Unlike other quantum algorithms, Grover's algorithm provides ``only" a quadratic speedup instead of exponential speedup over classical counterparts. However, even quadratic speedup is important and considerable, since search algorithms are widely used in various fields.











\begin{table}[!htb]
\fbox{\parbox{0.965\columnwidth}{\texttt{ 
function Grover(f,n):
\begin{enumerate}
    \item Using a Hadamard transform, prepare the equal $n$-qubit superposition of all $2^n$ logical basis states,
    \begin{align}
    \ket\psi_0 &= \hat{H}^{\otimes n}\ket{0}^{\otimes n}	 \nonumber \\
    &= \frac{1}{\sqrt{2^n}}\sum_{x = 0}^{2^n - 1}\ket{x}.
    \end{align}
    \item This can be equivalently re-expressed as,
    \begin{align}
    \ket\psi_0 &= \sqrt{\frac{M}{N}}\ket{T} + \sqrt{\frac{N-M}{N}}\ket{T_\bot},
    \end{align}
  	where $\ket{T}$ is the uniform superposition of the $M$ target elements, and $\ket{T_\bot}$ the uniform superposition of the remaining \mbox{$N-M$} non-target elements.
  	\item Apply the ???? operator,
  	\begin{align}
  		\hat{U}_T = \hat{I} - 2\ket{T}\bra{T}.	
  	\end{align}
	\item Apply the ????? operator,
	\begin{align}
		\hat{U}_\varphi = 2\ket\varphi\bra\varphi - \hat{I}.
	\end{align}
	\item Measure the $n$ qubits to obtain a sample from the target elements.
	\item ???$BOX$
\end{enumerate}
}}}
\caption{.} \label{alg:quant_search}
\end{table}



The algorithm is called single target Grover's algorithm if $M=1$, and called multiple targets Grover's algorithm if $M>1$. In addition, instead of searching the whole database for the target element, we could divide the whole database in several blocks, and then implement a variant of the algorithm, named partial search, to look for the block which contains the target element \cite{grover2005acm}. Subsequently, the partial search algorithm has been optimized \cite{choi2007quantum, korepin2005optimization, korepin2006quest, korepin2006group} and further generalized to hierarchical quantum partial search algorithm \cite{korepin2009quantum, korepin2007hierarchical}.

Beside the partial search, Grover's algorithm has been generalized to many other application scenarios, such as amplitude estimation \cite{brassard2002quantum}, quantum counting \cite{boyer1996tight,brassard1998quantum, mosca1998quantum}, finding the minimum \cite{durr1996quantum, nayak1999quantum, kowada2008new}, applying for arbitrary initial complex amplitude distributions \cite{biham1999grover}, fixed-point quantum search \cite{grover2005fixed,tulsi2005new,yoder2014fixed}, and multi-phase search \cite{tan2014quantum}. Furthermore, a closely related but more difficult problem, namely spatial search, has been also studied. The database of spatial search are some graph structure, and for some well-connected graphs, $O(\sqrt{N})$ time is still achievable \cite{childs2004spatial,chakraborty2016spatial,wong2016quantum,janmark2014global,meyer2015connectivity,wong2016spatial}. So far, the Grover's algorithm has been demonstrated with NMR \cite{chuang1998experimental}, trapped ion \cite{brickman2005implementation}, photonic \cite{walther2005experimental}, and superconducting hardware \cite{dicarlo2009demonstration}.

\section{Integer factorisation}

Today, the security of the most widely used public-key cryptography scheme, RSA scheme, is based the difficulty of factoring large numbers for classical computer. Thus, the problem of efficient factoring numbers has attracted widespread attention in the field of computer and information Science. However, there is still no no classical algorithm that can factoring numbers in polynomial time \cite{pollard1974theorems}. In 1994, a major breakthrough was made by Peter Shor that quantum computers could efficiently factor numbers in polynomial time \cite{shor1997polynomial,shor1994algorithms}, posing a serious threat to information security in business transactions on the Internet, such as e-commerce.

Suppose we want to factor $N$ using Shor's algorithm, for a randomly chose number ${a}$ ($0<a<N$) that is co-prime to $N$, Shor's algorithm can output the minimum integer $r$ that satisfies ${{a^r}{\kern 1pt} \bmod {\kern 1pt} N = 1}$.  From this period $r$, the prime factors of $N$ are given by the greatest common divisor (GCD) of ${{a^{r/2}}{\kern 1pt}  \pm 1}$ and ${N}$, which can be solved classically. The Shor's algorithm can be broken into the following simple steps:

1. Creat two registers, register 1 and register 2, which have  $n=2\left\lceil {\log _2^N} \right\rceil$ qubits and $m=\left\lceil {\log _2^N} \right\rceil$, respectively. Initialize the quantum register 2 to $|0,...,0,1\rangle$, and the quantum register 1 to

\[\frac{1}{{\sqrt {{2^n}} }}\sum\limits_{x = 0}^{{2^n} - 1} {|x\rangle }\]


2. Apply the modular exponential function $f(x) = {a^x}\bmod N$ on register 2 when register 1 is in state $|x\rangle$, and then obtain the state

\[\frac{1}{{\sqrt {{2^n}} }}\sum\limits_{x = 0}^{{2^n} - 1} {|x\rangle|{a^x}\bmod N\rangle }\]

Due to the quantum parallelism, the quantum computer will calculate the function $a^{x} \bmod n$ for all $x \in \{ 0,...,{2^n}\}$ in parallel by only one step.

3. Apply quantum Fourier transformation (QFT) on the register 1, yielding

\[\frac{1}{{{2^n}}}\sum\limits_{y = 0}^{{2^n} - 1} {\sum\limits_{x = 0}^{{2^n} - 1} {{e^{2\pi ixy/{2^n}}}|y\rangle |{a^x}\bmod N\rangle } } \]

4. Measure the registers. Due to the QFT in step 3, register 1 will output $|y\rangle$ with high probability, where $y = c{2^n}/r$ (for integer $c$), which could enable to deduce the period $r$ by some post processing on a classical computer. Then, the factor of $n$ can be determined by taking $\gcd(x^{r/2} + 1, n)$ and  $\gcd(x^{r/2} - 1, n)$.

Given an $n$-bit integer, the best rigorously proven upper bound on the classical complexity of factoring is $O({2^{n/4 + o(1)}})$ \cite{pollard1974theorems, strassen1976einige}. However, the Shor's algorithm can solve this task in $O({n^3})$ time \cite{shor1997polynomial,shor1994algorithms}, which achieves an exponential speedup over all classical algorithms.  Furthermore, for some special case, such as factoring safe semiprimes (an important class of numbers used in cryptography), more feaster quantum algorithm is developed by changing the classical part of Shor's algorithm \cite{grosshans2015factoring}. In the experimental side, some proof-of-principle experiments with a small number of qubits have already been achieved \cite{monz2016realization, lucero2012computing, martin2012experimental, politi2009shor, lu2007demonstration, lanyon2007experimental, vandersypen2001experimental, johansson2017realization, huang2017experimental}.


\section{Topological data analysis}

Topological data analysis (TDA) \cite{carlsson2009topology} is a tool for extracting useful information from unstructured data by studying the shape of data. In particular, it can be used to estimate the certain topological features of data, such as the Betti numbers, which count the number of holes and voids of various dimensions in a scatterplot. In 2016, Lloyd, Garnerone, and Zanardi proposed a quantum algorithms for TDA, quantum TDA, which can provide an exponential speedup over the best currently known classical algorithms \cite{lloyd2016quantum}.

Generally, the $k$-th Betti number refers to the number of $k$-dimensional holes. Consider $n$ data points, we will introduce the steps of the quantum TDA algorithm for calculating the $k$-th Betti number:

1. Simplicial complex construction. Implement the Grover's algorithm with a membership oracle function $\{f_k^\epsilon ({s_k}) = 1$ if ${s_k} \in S_k^\epsilon \}$ to construct the simplicial complex state,

\[|\psi \rangle _k^\epsilon  = \frac{1}{{\sqrt {|S_k^\epsilon |} }}\sum_{{s_k} \in S_k^\epsilon } {|{s_k}\rangle }.\]
where the $k$-simplex $|s_k\rangle$ is an $n$-qubit quantum state with ${k+1}$ 1s at positions ${{j_0},{j_1},...,{j_k}}$ and 0s at the other remaining positions, which means a connected graph with points ${{j_0},{j_1},...,{j_k}}$, and the Vietoris-Rips simplicial complex ${S_k^\epsilon }$ is the set of ${k}$-simplices where all points are within distance ${\epsilon }$ of each other.

2. Mixed state construction. Add an ancillary register consisted of $n$ qubits and then perform controlled-NOT (CNOT) gates to copy $|\psi \rangle _k^\epsilon$ to construct ${\frac{1}{{\sqrt {|S_k^\epsilon |} }}\sum\limits_{{s_k} \in S_k^\epsilon } {|{s_k}\rangle }  \otimes |{s_k}\rangle }$, finally trace out the ancillary register to obtain mixed state ${\rho _k^\epsilon }$.

\[{\rho _k^\epsilon  = \frac{1}{|S_k^\epsilon |}\sum_{{s_k} \in S_k^\epsilon } {|{s_k}} \rangle \langle {s_k}{\rm{|}}}.\]

3. Topological analysis. Define the boundary map ${\partial _k^\epsilon }$ as $\partial _k^\epsilon |{s_k}\rangle  = \sum\limits_l {{{( - 1)}^l}} |{s_{k - 1}}(l)\rangle$, where ${|{s_{k - 1}}(l)\rangle }$ is obtained from ${{s_k}}$ with vertices ${{j_0}...{j_l}...{j_k}}$ by omitting the ${l}$-th point ${{j_l}}$ from ${{s_k}}$. Then, transform the boundary map to a Hermitian matrix

\[B_k^\epsilon  = \left( {\begin{array}{*{20}{c}}0&{\partial _k^\epsilon }\\{\partial {{_k^\epsilon }^\dag }}&0\end{array}} \right).\]

Apply the phase-estimation algorithm to decompose ${\rho _k^\epsilon }$ in terms of the eigenvectors and eigenvalues of ${B_k^\epsilon }$, and then measure the eigenvalue register. Employing the  probability of measuring zero ${\eta _k^\epsilon }$, the dimension of the kernel of ${\partial _k^\epsilon }$ could be calculated as ${\dim({\rm{Ker}}\,\partial _k^\epsilon) = \eta _k^\epsilon  \cdot |S_k^\epsilon |}$. Using the similar approach for calculating ${\dim({\rm{Ker}}\,\partial _{k + 1}^\epsilon )}$, then we can reconstruct the ${k}$-th Betti number by,

$
\beta _k^\epsilon =\dim({\rm{Ker}}\,\partial _k^\epsilon)  - \dim({\mathop{\rm Im}\nolimits} \,\partial _{k{\rm{ + }}1}^\epsilon) \nonumber=\dim({\rm{Ker}}\,\partial _k^\epsilon)  + \dim({\rm{Ker}}\,\partial _{k{\rm{ + }}1}^\epsilon) - |S_{k + 1}^\epsilon |.
$

Above is the basic idea of calculating the $k$-th Betti number. In practical application, we can define the full Hermitian boundary map, Dirac operator, to be ${B^\epsilon } = B_1^\epsilon  \oplus B_2^\epsilon  \oplus ... \oplus B_n^\epsilon$, and then use the Dirac operator to estimate Betti numbers to all orders (see \cite{lloyd2016quantum} for details).

In theory, the quantum TDA algorithm could estimate approximate values of Betti numbers to all orders and to accuracy $\delta$ in time $O({n^5}/\delta )$ \cite{lloyd2016quantum}, by contrast, the best classical algorithms for estimating Betti numbers to all orders to accuracty $\delta$ takes time at least $O({2^n}log(1/\delta ))$ \cite{cohen2007stability,basu1999bounding,*basu2003different,*basu2008computing,*basu2014algorithms,friedman1998computing}.

\bibliography{algorithm}

\end{document}
%
% ****** End of file apstemplate.tex ******
