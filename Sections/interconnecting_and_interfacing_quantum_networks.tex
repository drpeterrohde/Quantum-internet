%
% Interconnecting & Interfacing Quantum Networks
%

\section{Interconnecting \& interfacing quantum networks} \label{sec:inter} \index{Interfacing quantum networks}

Any global-scale network will inevitably comprise participants choosing to go about things their own way. The physical architecture and medium may vary from one subnetwork to the next, as may the QTCP policies they adopt. The key then is to construct efficient \textit{interconnects} between different levels of the network hierarchy, each of which may subscribe to their own QTCP policies and cross between different physical mediums. Note that the QTCP protocol presented here does not enforce any particular networking policies, but rather provides a high-level framework that can be customised essentially arbitrarily.

For example, the cost metrics and attributes employed at the intercontinental level would most certainly be very different to those in a small LAN. A small LAN might be running applications whereby they can easily reproduce packets and thereby tolerate packet loss. But for a warehouse-scale commercial quantum computing enterprise, responsible for performing one stage of a distributed quantum computation, the loss of a single packet could be extremely costly, requiring the entire computation to be performed completely from scratch due to no-cloning and no-measurement limitations, something that may not come cheaply.

Such interconnects will typically comprise a combination of:
\begin{itemize}
\item Packet switching\index{Packet switching}: such that packets can be arbitrarily switched between the different levels of the network hierarchy.
\item Physical interface: interconnect may be switching between different media. Such physical interfaces have costs associated with them. For example, coupling between free-space and fibre is typically very lossy. Sec.~\ref{sec:opt_inter} discusses optical interfacing with matter qubits, and Sec.~\ref{sec:hybrid} discusses hybrid architectures, where optics mediates entanglement generation between matter qubits.
\item Quantum memory\index{Quantum memory}: such that data can be buffered while it awaits its turn at being switched between networks, as different networks may have different loads and operate at different clock-rates. This is discussed in Sec.~\ref{sec:memory}.
\item Packet format conversion\index{Packet format conversion}: different levels of the network hierarchy may be employing entirely different cost metrics, attributes, and cost functions, requiring packet headers to be reformatted upon switching between networks.
\end{itemize}

The packet switching and quantum memory are implemented as quantum processes at nodes, using the usual quantum process formalism. The physical interface between different mediums, if there is one, could be very diverse, encompassing many types of physical systems, but can always be characterised using the quantum process formalism. Packet headers, which contain all formatting, cost, and routing information are represented entirely classically and communicated entirely by the classical network. Thus, this operation also takes place at nodes, but no quantum processes are taking place.