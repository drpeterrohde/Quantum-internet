%
% How to learn more & get your hands dirty
%

\section{How to learn more \& get your hands dirty}\index{Learn more}

\dropcap{F}{or} the reader interested in delving deeper into quantum technology with a hands-on approach, there are a number of freely available software tools for simulating various quantum technologies, including quantum computing and quantum networks. We summarise some of the major ones below, all of which are freely available:

\begin{itemize}
	\item SimulaQron (\href{http://www.simulaqron.org}{http://www.simulaqron.org}): Developed by QuTech Delft, a simulator for designing software for the quantum internet.
	\item Google Cirq (\href{https://github.com/quantumlib/Cirq}{https://github.com/quantumlib/Cirq}): Google's Python-based SDK for designing quantum software, specifically targeted at development for Google's own in-house quantum computing hardware platform.
	\item Microsoft Quantum Development Kit and the Q\# language (\href{https://www.microsoft.com/en-au/quantum/development-kit}{https://www.microsoft.com/en-au/quantum/development-kit}): a full software development kit for writing code executable on quantum computers. The Q\# language is used to describe quantum operations, which classical code written in C\# can interface with and control. The platform is extremely versatile and allows highly complex quantum operations to be coded in a high-level, platform-independent manner.
	\item IBM Quantum Experience (\href{https://quantumexperience.ng.bluemix.net}{https://quantumexperience.ng.bluemix.net}): IBM's cloud-based platform for remotely executing simple quantum protocols, as written by the user. The platform includes QISKit, a Python-based SDK for executing quantum algorithms.
	\item \comment{What else?}
\end{itemize}
