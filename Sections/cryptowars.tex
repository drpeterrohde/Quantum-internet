%
% CryptoWars (TM)
%

\section{CryptoWars\texttrademark}\index{CryptoWars}

\comment{To do, AES, DES, RSA}

\comment{Index}

Undoubtedly, quantum technologies will be most impactful (and disruptive) in the area of information security, something of fundamental importance to all of us on a daily basis. Quantum technologies will be important both in terms of breaking and maintaining security, with the former mandating interest in the latter.

In Sec.~\ref{sec:homo_blind} we discussed encrypted outsourced quantum computation as an important concept in future cloud quantum computing. In this section we will step back from full-fledged distributed quantum computation, instead focussing on more elementary protocols for simple two-party secure communication, the most foundational cryptographic primitive.

\subsection{The end of classical cryptography?}

One of the main reasons quantum computation has received so much interest from nation states is their ability to efficiently crack some cryptographic protocols, most notably RSA.

\subsubsection{Private-key cryptography}\index{Private-key cryptography}

Private- (or symmetric-) key cryptography is perhaps the most basic (and useful) cryptographic primitive, enabling encryption of a channel between two parties who share a secret key. The same secret key is employed for both encryption and decryption operations, making it of utmost importance that it be retained secret.

Private key cryptography has a long history, in fact going back to ancient times, enabling the secret sharing of diplomatic messages between statesmen. However it was a niche technology that very few utilised, since it had to be implemented by hand without computers. Today, however, the ability to communicate secretly with others is completely taken for granted in all but a few nations and resides in every smartphone.

Today there are countless freely available private-key cryptographic protocols available online. 

\comment{Note that when performing a brute-force attack against a private encryption key\index{Private-key encryption}, a quadratic speedup effectively halves the key length in terms of algorithmic runtime. Thus, in the quantum era private key lengths will need to be doubled.}

\comment{Shift discussion of halving private keylength by half from footnote in early section.}

\subsubsection{Public-key cryptography}\index{Public-key cryptography}

While private-key cryptography solves the problem of end-to-end cryptography, it has one main downfall -- how does one share a private key between two parties? After all, if we had the ability to secretly share keys between ourselves, wouldn't we just use that method to directly communicate, bypassing the unnecessary cryptographic protocol.

Public- (or asymmetric-) key cryptography addresses this issue by replacing the private key with two keys, one used solely for \textit{encryption}, the other solely for \textit{decryption}. Then to send a message to a friend I can send him my encryption (public) key that he is only able to use for preparing an encrypted message for me. No security is required when sharing the public key since an eavesdropper can't use it for decryption. Finally, I am able to decrypt the message using my decryption (private) key, which I kept completely to myself and never shared with anyone.

RSA \cite{bib:RSA} was the first published public-key cryptographic protocol.

\subsubsection{The Blockchain}\index{Blockchain}

\subsection{Quantum cryptography}\index{Quantum cryptography}

\subsection{Post-quantum classical cryptography}\index{Post-quantum classical cryptography}

\comment{Mcelise, NP-hard problem}

\subsection{The quantum Blockchain}\index{Quantum Blockchain}