%
% Quantum Networks
%

\dropcap{W}{e} have reviewed some of the key aspects of classical networks, including the real-world implementation of classical networking via the TCP/IP protocol stack, and the essential mathematical foundations for networking theory, including cost vector analysis and routing strategies.

Let us now lay the foundations and some of the key motivations and assumptions we will make in our upcoming discourse on quantum networks, and lay out some of the key differences between classical networks and future quantum ones.

Quantum networks comprise all the same ingredients as classical networks, but with some very important non-classical additions. Nodes can additionally implement quantum computations, quantum-to-classical interfaces (i.e measurements), quantum-to-quantum interfaces (i.e switching data between different physical systems), quantum memories, or any quantum process in general. Many of these are not allowed by the laws of classical physics.

The cost vectors associated with links could include measures that are uniquely quantum, such as fidelity, purity or entanglement measures, none of which are applicable to classical digital data.

As in the classical case, our goal is to find routing strategies that optimise a chosen cost measure. But in the quantum context costs will be constructed entirely differently owing to the quantum nature of the information being communicated.

We envisage a network with a set of senders and receivers, all residing on a time-dependent network graph as before. Senders have sets of quantum states they wish to communicate. For each state they must choose appropriate strategies, such that the overall cost is optimised, for some appropriate cost measure. Compared to classical resources, equivalent quantum resources are costly and must be used efficiently and frugally. Indeed, the no-cloning theorem\index{No-cloning theorem} imposes the constraint that arbitrary unknown states cannot be replicated at all! This makes resource allocation strategies of utmost importance in the quantum world.

Routing strategies will not always guarantee that packets have immediate access to network bandwidth the moment they demand it. One needs to think about the others too! Inevitably, in shared networks there will sometimes be competition and congestion, forcing some users to wait their turn. For this reason, many quantum networks will require at least some nodes (the ones liable to competition) to have access to quantum memories, such that quantum packets can be buffered for a sufficient duration that they can wait their turn on the shared network resources for which there is high competition. The required lifetime of a quantum memory will then be related to overall network congestion. Of course, quantum memories induce unwanted quantum processes of their own, which need to be factored into cost calculations.

Given that classical networking is decades more advanced than quantum networking, and extremely cheap and reliable in comparison, we will assume that classical resources `come for free', and only quantum resources are of practical interest in terms of their cost. That is, classical communication and computation is a free resource available to mediate the operation of the quantum network. We therefore envisage a \textit{dual network}\index{Dual network} with two complementary networks operating in parallel and in tandem -- the quantum network for communicating quantum data, and a topologically identical classical network operating side-by-side and synchronised with the quantum network, overseeing and mediating the quantum network.

Data packets traversing the network will comprise both quantum and classical fields, which will be separated to utilise the appropriate network, but synchronised such that they arrive at their destination as a single package of joint quantum and classical information to be at the disposal of the recipient.

The motivation for the dual network is to ensure that classical and quantum data that jointly represent packets remain synchronised and subject to the same QoS issues, such as packet collisions and network congestion.

We envisage quantum networks to extend beyond just client/server quantum computation, to include the free trade of any quantum asset. This includes state preparation, measurement, computation, randomness, entanglement, and information. Much like the classical internet, by allowing quantum assets to be exchanged, we can maximise utility, improve economy of scale, and enable new models for commercialisation.\index{Quantum assets}

May the games begin.