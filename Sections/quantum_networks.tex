%
% Quantum Networks
%

\section{Quantum networks} \label{sec:quant_net} \index{Quantum networks}

\comment{Expand or make essay}

Quantum networks comprise all the same ingredients as classical networks, but with additions. Nodes can additionally implement quantum computations, quantum-to-classical interfaces (i.e measurements), quantum-to-quantum interfaces, quantum memories, or any quantum process in general. The cost associated with links could include measures that are uniquely quantum, such as fidelity, purity or entanglement measures, none of which are applicable to classical digital data. As in the classical case, our goal is to find routing strategies that optimise a chosen cost measure. But in the quantum context costs will be constructed entirely differently owing to the quantum nature of the information being communicated.

We envisage a network with a set of senders, $\{A_i\}$, and receivers, $\{B_i\}$, residing on a time-dependent network graph, $G_t$, as before. $A$ have sets of quantum states to communicate, $\{\hat\rho_\text{data}(i)\}$. For each $\hat\rho_\text{data}(i)$ she must choose appropriate strategies $S_t$, such that the overall cost is optimised, for some appropriate cost measure. Compared to classical resources, equivalent quantum resources are costly and must be used efficiently, making resource allocation strategies of utmost importance.

The routing strategies we introduce in Sec.~\ref{sec:strategies} will not always guarantee that packets have immediate access to network bandwidth the moment they demand it. Inevitably, in shared networks there will sometimes be competition and congestion, forcing some users to wait their turn. For this reason, many quantum networks will require at least some nodes (the ones liable to competition) to have access to quantum memories, such that quantum packets can be stored for a sufficient duration that they can wait their turn on the shared network resources. The required lifetime of a quantum memory will then be related to overall network congestion. Of course, quantum memories induce quantum processes of their own, which need to be factored into cost calculations. Quantum memories are easily accommodated for in the QTCP framework by allowing self-loops at nodes, which implement a memory process. The implementation of quantum memory is discussed in Sec.~\ref{sec:memory}.

Given that classical networking is decades more advanced than quantum networking, and extremely cheap and reliable, we will assume that classical resources `come for free', and only quantum resources are of practical interest. That is, classical communication and computation is a free resource available to mediate the operation of the quantum network. We therefore envisage a \textit{dual network} with two complementary networks operating in parallel and in tandem -- the quantum network for communicating quantum data, and a topologically identical classical network operating side-by-side and synchronised with the quantum network.

The motivation for the dual network is to ensure that classical and quantum data that jointly represent packets (Sec.~\ref{sec:prot_stack}) remain synchronised and subject to the same QoS issues, such as packet collisions (Sec.~\ref{sec:collision}) and network congestion.

We envisage quantum networks to extend beyond just client/server quantum computation, to include the free trade of any quantum asset. This includes state preparation, measurement, computation, randomness, entanglement, and information. Much like the classical internet, by allowing quantum assets to be exchanged, we can maximise utility, improve economy of scale, and enable new models for commercialisation.\index{Quantum assets}