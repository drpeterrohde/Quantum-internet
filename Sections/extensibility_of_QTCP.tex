%
% Extensibility of QTCP
%

\subsection{Extensibility of QTCP}\index{Extensibility of QTCP} \label{sec:c_vs_a}

In Sec.~\ref{sec:costs} we introduced the notion of the \textit{costs} and \textit{attributes} of links in a classical network, and in Sec.~\ref{sec:quantum_meas_cost} generalised these notions to the quantum case. In Sec.~\ref{sec:packet_header} we described the header format for quantum data packets.

The \textsc{Costs} and \textsc{Attributes} fields within the \textsc{Header} are very powerful data structures, implemented as ordered sets of arbitrary dimension, comprising arbitrary data fields. The intention here is to allow QTCP to be extensible into the future, with the flexible addition of new data structures into the protocol. These can be custom designed to, in conjunction with appropriate routing strategies and cost functions, influence the operation of QTCP completely arbitrarily, and easily implement entirely different quantum networking paradigms than presented here.

\textsc{Costs} naturally capture characteristics of the network that accumulate additively along routes, whereas \textsc{Attributes} capture any other characteristics that aren't additive. A network needn't have both costs \textit{and} attributes. It may have one or the other, or both, but not neither, since there must be some measure by which to judge routes, even if via a very trivial measure.

\comment{To do!}