%
% State-of-the-Art
%

\startnormtable

\dropcap{W}{hen} will we have the quantum internet? This is a question with as many answers as there are people one asks. And the developments of the different components of such an internet will be staggered and under continual development -- a quantum internet with the capacity for long-distance QKD will likely arrive far sooner than one supporting fully distributed, blind quantum computation.

In this section we discuss recent developments in and the state of the art of some of the most important quantum technologies and protocols, with a view to understand trends in the development of the field, so as to gauge the rate of progress. With this we aim to shed light on what quantum networking services could be readily implemented today, using present-day technology, and what is likely to be viable in the near future.

We will very succinctly summarise a brief history of major recent developments in the field, without much elaboration into the specific developments. Rather our goal here is simply to very tersely summarise recent progress and the state of the art. We provide detailed referencing so that the interested reader can follow up on the specifics of these important developments of interest. The reader disinterested in a history lesson might skip this section.

%
% Quantum Teleportation & Entanglement Distribution
%

\section{Quantum teleportation \& entanglement distribution} \index{Quantum state teleportation} \index{Quantum gate teleportation} \index{Entanglement distribution}

\dropcap{Q}{uantum} state teleportation has attracted broad experimental interest and been subject to widespread demonstration across many physical architectures, becoming one of the most well investigated protocols in experimental quantum information science. Table.~\ref{tab:state_tomo} summarises some of the notable developments.

\begin{table*}[htpb]
\caption{Developments in experimental quantum state teleportation and entanglement distribution.} \label{tab:state_tomo}
\begin{tabular}{|p{0.755\linewidth}|p{0.22\linewidth}|}
	\hline
	Summary & References \& years \\
	\hline \hline
	A technique for the generation of high-intensity polarisation-entangled photon pairs. For the partial Bell state projection a 50:50 beamsplitter was employed. & \cite{kwiat1995new, bib:Euro_25_559} \\
	\hline
	The first experimental demonstration of photonic quantum teleportation. Two photon-pairs were prepared by double-pumping a single non-linear beta-barium borate (BBO) crystal: one pair employed as the entanglement source; the other to prepare the state to teleport. The partial Bell state measurement was implemented using which-path erasure at a beamsplitter, with an efficiency of 25\%. & \cite{bib:Boumeester97} \\
	\hline
	Experimental demonstration of quantum state teleportation between two labs, separated by 55m, but connected by a 2km length of fibre, with photons at telecommunication wavelengths. This arouses the exciting prospect that future quantum networks might be able to piggyback off existing telecom infrastructure, which would be a boon to the quantum industry. & \cite{bib:Nat_421_509} \\
	\hline
	Quantum state teleportation was demonstrated between photonic and atomic qubits, a first step towards hybrid architectures, and an essential ingredient in interfacing optical and non-optical systems. & \cite{bib:Chen08} \\
	\hline
	Quantum teleportation over a 16km long, noisy, free-space channel between distant ground stations was demonstrated. This distance is of especial interest as it is significantly longer than the effective thickness of the atmosphere, equivalent to 5-10km of ground atmosphere. This is an exciting benchmark as it suggests that free-space ground-to-satellite teleportation may be viable. & \cite{bib:Nat_Phot_4_376, bib:PRL_94_150501} \\
	\hline
	The teleportation distance in free-space was extended to 97km over Qinghai Lake, and 143km between the two Canary Islands of La Palma and Tenerife. These overcame the daunting challenges associated with source targeting and tracking, for long-distance, free-space quantum teleportation, and paved the way for future satellite-based quantum teleportation. & \cite{bib:Nat_488_185, bib:Nat_489_269} \\
	\hline
	Accompanying the breakthrough of superconducting single-photon detectors with near-unit efficiency, 3-fold photo-detection for quantum teleportation was greatly enhanced by more than two orders of magnitude at telecom wavelengths, and the teleportation distance in optical fibre lengthened to 100km. & \cite{bib:Optica_2_832} \\
	\hline
	Quantum teleportation over fibre networks in Hefei and Calgary were demonstrated, with lengths of dozens of kilometres. & \cite{bib:Nat_Phot_10_671, bib:Nat_Phot_10_676} \\
	\hline
	The first quantum satellite for entanglement distribution was launched in China. In addition to teleportation, this could facilitate intercontinental QKD. The team is aiming to achieve quantum teleportation between ground stations and satellite, and even between pairs of distant ground stations, separated by over 1000km, using shared entanglement provided by the satellite. & \cite{bib:Nat_535_478} \\
	\hline
	Using 5-photon entanglement, open-destination teleportation was implemented, whereby an unknown quantum state was teleported onto a superposition of 4 destination photons, which could be read out at any location -- a type of `broadcasting'. & \cite{bib:Nat_430_54} \\
	\hline
	The state of a two-photon composite system was demonstrated -- a breakthrough in the teleportation of a single particle onto a complex system comprising multiple particles. & \cite{bib:Nat_Phys_2_678} \\
	\hline
	Quantum teleportation over multiple degrees of freedom of a single optical mode was demonstrated. & \cite{bib:Nat_518_516} \\
	\hline
	Teleportation of CV optical states was demonstrated. The advantage of this teleportation protocol was that it could be deterministic in principle,  overcoming the non-determinism inherent to partial Bell state projections using a PBS. & \cite{bib:Science_282_706} \\
	\hline
	The deterministic teleportation of photonic qubits was demonstrated using hybrid techniques. & \cite{bib:Nat_500_315} \\
	\hline
	Quantum teleportation has also attracted great interest in other physical architectures. Demonstrations have been performed in various physical systems, including atoms, ions, electrons, and superconducting circuits. & \cite{bib:Nat_Phys_9_400, bib:Nat_429_734, bib:Nat_429_737, bib:Science_345_532, bib:Nat_500_319} \\
	\hline
Hybrid schemes combining different physical systems have been demonstrated, such as light-to-matter teleportation. Hybrid technologies are expected to play an important role in future quantum networks, where the underlying physical architecture for (say) a quantum computation is non-optical, but optics mediates the communication of quantum information. & \cite{bib:Nat_443_557, bib:Nat_Comm_4_2744} \\
	\hline
\end{tabular}
\end{table*}

%
% Entanglement Swapping & Quantum Repeaters
%

\section{Entanglement swapping \& quantum repeaters} \index{Entanglement swapping} \index{Quantum repeater networks}

\dropcap{C}{losely} related to entanglement distribution is entanglement swapping (Sec.~\ref{sec:swapping}), where the goal is to entangle two remote parties, each of whom have one half of two distinct Bell pairs. As discussed in Sec.~\ref{sec:ent_ultimate}, entanglement swapping will be of fundamental importance in networks treating entanglement distribution as the most fundamental elementary resource, making it of great importance to quantum networking, allowing the distribution of entanglement between distant parties, who are unable to communicate directly (i.e the entanglement distribution is `mediated').

The first experimental demonstration of entanglement swapping was presented in 1998 \cite{bib:PRL_80_3891}. By pumping a BBO crystal in a double-pass configuration, two pairs of polarisation-entangled photons were generated to demonstrate the scheme. A visibility of \mbox{$0.65$} was observed, which clearly surpasses the classical limit of \mbox{$0.5$}. This was later improved in 2001 to a visibility of \mbox{$0.84$} \cite{bib:PRL_86_4435}, which violates the Bell inequality (for which the threshold is $0.71$). 

Aside from `event-ready' mode, \cite{bib:JMO_47_2} proposed a delayed-choice mode of operation for entanglement swapping in 2000, where entanglement is produced a posteriori, after the entangled particles have been measured and may no longer even exist.

In 2001, \cite{bib:PRL_88_017903} designed and realised delayed-choice entanglement swapping. This was performed by adding two 10m optical fibre delays of about 50ns for both outputs of the Bell state measurement. Subsequently, in 2002 \cite{bib:PRA_66_024309} realised a delayed-choice entanglement swapping experiment with vacuum and one-photon quantum states. However, none of these demonstrations were active, random or delayed choice, which are required to ensure that photons cannot know in advance the basis choices for future measurements.

In 2012, \cite{bib:Nat_Phys_8_479} demonstrated an entanglement swapping experiment with active delayed choice. In their experiment, they designed a special interferometer to realise active switching between Bell state measurement and separable state measurement, and experimentally verified the entanglement-separability duality of the two photons. 

Subsequently, experimental demonstrations of entanglement swapping have evolved to become more complex and rigorous, finding uses in more sophisticated networking protocols.

In 2008, \cite{bib:PRL_101_080403} used three pairs of polarisation-entangled photons, and conducted two Bell state measurements to realise multistage entanglement swapping. \cite{bib:PRL_103_020501} demonstrated multi-particle entanglement swapping using a three-photon GHZ state. In 2013, \cite{bib:PRL_110_210403} demonstrated entanglement swapping between photons that never coexisted. In their experiment, entangled photons are not only separated spatially, but also temporally. In 2015, \cite{bib:PRL_114_100501} experimentally realised hybrid entanglement swapping between discrete- and CV optical systems.

To develop a practical quantum network, entanglement swapping between independent entangled photon sources is a important. In the past two decades, entanglement swapping has been demonstrated in a large number of experiments across many physical architectures. However, in most experiments, entangled photons are generated by using the same laser, and therefore do not meet the requirements for independence. Entanglement swapping based on independent entangled photon sources has been experimentally verified \cite{bib:PRL_96_110501, bib:Nat_Phys_3_692, bib:PRA_79_040302}, but the distinguishability caused by photon propagation in the channel is still a great obstacle to realising entanglement swapping using independent sources under realistic conditions.

In 2015, \cite{bib:Nat_526_682} for the first time achieved entanglement swapping using independent entangled photon sources separated by 1.3km in a real-world environment. However, the wavelength of the photons used in this experiment was 637nm (transmission loss \mbox{$\sim 15$dB/km}), which is not conducive to achieving long-distance entanglement swapping since it is far greater than the transmission loss of communication-band photons in fibre (\mbox{$\sim 0.2$dB/km}). Based on the techniques in \cite{bib:Nat_phot_10_671, bib:Nat_phot_10_676}, longer-distance entanglement swapping may be possible in the near future.

Entanglement swapping can also be directly used for QKD. Alice and Bob each have an entangled photon source, and one photon of each Bell pair is sent to a third-party measurement node, Eve. Similar to measurement-device-independent (MDI) QKD, the security of the generated private key does not depend on Eve's faithful execution of the operation. That is, Eve can be an untrusted third party. This MDI property also reflects the physical beauty of quantum teleportation. Bell state measurements do not reveal any information about the quantum state, but can be used to restore the transmitted quantum state. On the other hand, quantum entanglement occurs between the remaining photons in the Bell pair of Alice and Bob. \cite{bib:PRL_90_057902, bib:NJP_10_2008} suggest that an entangled photon source can be considered as a basis-independent light source for QKD. Thus, the QKD realised by entanglement swapping has the characteristics of MDI and light source independence.

An important application for entanglement swapping is that we can entangle spatially separated and independent matter qubits by coupling them with photons, upon which entanglement swapping is subsequently applied. This is an extremely important technique for hybrid quantum networks (Sec.~\ref{sec:hybrid}), where optical interactions mediate entanglement swapping between non-optical qubits.

Starting with two entangled atom-photon pairs, we can project the two atomic qubits into a maximally entangled state by performing a Bell state measurement on the two photons \cite{bib:Nature_428_153, bib:PRL_96_030404}. In 2007, \cite{bib:Nature_449_68} entangled two trapped atomic ions separated 1m apart using entanglement swapping, exploiting interference between photons emitted by the ions. The fidelity of the states of the entangled ions was $0.63(3)$. In subsequent experiments \cite{bib:PRL_100_150404}, the ion-ion entanglement fidelity was improved to $0.81$. Similarly, \cite{bib:Nature_454_1098} entangled two atomic ensembles, each originally with a single emitted photon, by performing a joint Bell state measurement on the two single photons after they had passed through a 300m fibre-based communication channel. In 2015, \cite{bib:Nature_526_682} generated robust entanglement (estimated state fidelity of $0.92\pm0.03$) between the two distant spins by entanglement swapping in the scheme of \cite{bib:PRA_71_060310, bib:Nature_497_86}. Such a high fidelity is sufficient to successfully perform loophole-free Bell inequality tests.

Entanglement swapping is a core element of quantum repeaters, which is of great significance to realising long-distance quantum communication. At present, the maximum transmission distance that can be achieved by QKD is 400km \cite{bib:arxiv_1606.06821}. Therefore, the communication distance is still a bottleneck restricting the development of quantum communication. Quantum repeaters, proposed in 1998 \cite{bib:PRL_81_5932}, combine entanglement swapping and quantum memory, which provides a potential solution to this problem.

The first proposed practical quantum repeater architecture was proposed in 2001 by Duan, Lukin, Cirac \& Zoller (DLCZ) \cite{bib:DLCZ}, using atomic ensembles and linear optics. To increase the repeater count-rate, various protocols \cite{bib:RMP_83_33, bib:PRA_79_042340, bib:PRA_92_012307, bib:PRA_81_052311, bib:PRA_81_052329, bib:NP_6_777, bib:PRL_112_250501} have been proposed.

In 2015, \cite{ncomms7787} introduced the concept of all-photonic quantum repeaters, based on flying qubits, which entirely mitigate the need for a matter quantum memory. 

Experimental demonstration of elementary segments of quantum repeaters were achieved by \cite{bib:Sc_316_1316, bib:Nat_454_1098}.

In order to develop practical quantum repeaters, there are many experimental techniques that must be developed, such as the multiplexing technique \cite{bib:PRA_76_050301, bib:PRA_82_010304, bib:PRL_113_053603, bib:PRL_98_060502} for constructing multimode memories. Techniques based on non-degenerate photon-pair sources \cite{bib:Nat_469_508, bib:Nat_469_512, bib:PRL_112_040504, bib:PRA_92_012329} and quantum frequency conversion \cite{bib:NP_6_894, bib:NC_5_3376} are being developed to obtain quantum memories compatible with photons at telecom wavelengths.

Aside from photonic systems, techniques based on other physical systems have also been developed \cite{bib:NP_11_37, bib:Sc_337_72, bib:N_484_195, bib:N_497_86}. In general, to enable scaling up to repeaters with several links, many techniques need to be considerably improved and simplified, and it appears there is still a long way to go before building a first practical, long-distance quantum repeater.

%
% Quantum Key Distribution
%

\section{Quantum key distribution} \index{Quantum key distribution (QKD)}\ref{bib:QKD_state_of_art}

\dropcap{W}{ith} the development of quantum technology, QKD will gradually become more and more practical and economically accessible. Bennett, one of the proposers of the BB84 protocol, first demonstrated the protocol on an optical platform with a distance of 30cm \cite{bib:JC_5_3}. Since then, experiments have developed rapidly from indoors to outdoors, over ever-increasing distances, with commercial QKD units even available as off-the-shelf products. Table.~\ref{tab:QKD_table} summarises some major developments in the field.

\begin{table*}[htpb]
\caption{Developments in experimental QKD.} \label{tab:QKD_table}
\begin{tabular}{|p{0.755\linewidth}|p{0.22\linewidth}|}
	\hline
	Summary & References \& years \\
	\hline \hline
	Quantum cryptography using polarised photons in optical fibre over more than 1km. & \cite{bib:EL_23_383} \\
	\hline
	QKD experiment over 10km using phase-encoding. & \cite{bib:EL_29_634} \\
	\hline
	Outdoor experiment over 67km using a plug-and-play system to automatically maintain stabilisation. & \cite{bib:Arx0203118} \\
	\hline
	QKD based on decoy-states over more than 100km, marking the beginning of long-distance QKD. & \cite{bib:PRL_98_010505, bib:PRL_09_010503x, bib:PRL_98_010504} \\
	\hline
	Decoy-state QKD over a 200km optical fibre cable through photon polarisation with a final key rate of 15Hz & \cite{bib:OptExp_18_8587} \\
	\hline
	First realisation of a differential phase-shift (DPS) QKD protocol over a 42.1dB lossy channel and 200km of optical dispersion-shifted fibre. & \cite{bib:NP_1_343} \\
	\hline
\end{tabular}
\end{table*}

\comment{Up to here}

In 2012, \cite{bib:OL_37_1008} realised the DPS protocol over 50dB channel loss and 260km of optical fibre using superconductive detectors, this is the first implementation of QKD over more than 50dB channel loss.

In 2009, \cite{bib:NJP_11_075003} realised the coherent one way (COW) protocol QKD system with a maximum range of 250km at 42.6dB channel loss using ultra-low-loss fibre, with secret bit rates up to 15Hz.
 
Apart from using the QKD scheme based on state preparation and measurement, schemes based on entanglement distribution, mainly the E91 \cite{bib:PRL_67_661} and BBM92 \cite{bib:PRL_68_557} protocols, have been demonstrated, which are also undergoing extensive experimental investigation.

In 2005, \cite{bib:OE_13_202} distributed entanglement and single photons through over a free-space quantum channel, demonstrating the viability of free-space quantum communication. 

In 2006, \cite{bib:APL_89_101122} reported a complete experimental implementation of a QKD protocol over a free-space link using polarisation-entangled photon pairs.

In 2007, \cite{bib:NP_3_481} realised the BBM92 QKD protocol based on polarisation encoding over 144km.

The experiments listed above indicate that QKD protocols based on free-space entanglement distribution have the advantage of being less affected by decoherence, which lay a solid foundation for global and satellite-to-ground quantum communication.

Fibre loss increases exponentially with distance. However, the loss of free-space transmission increases very little with distance,  mainly related to the thickness of the atmosphere. Therefore, it is a perfectly reasonable solution to construct the global quantum internet based on satellite communication. To verify the feasibility of a quantum channel between space and Earth, a European Union group successfully received weak light pulses emitted from a ground station and reflected by a mirror placed on a low-orbiting satellite with orbital altitude of 1485km in 2008 \cite{bib:NJP_10_033038}. In the context of rapidly moving platforms, \cite{bib:NP_7_382} realised QKD over 20km from an airplane to the ground in 2013. In the same year, \cite{bib:NP_7_387} successfully accomplished quantum communication with a hot-air balloon floating platform. The experiments on aeroplanes and hot-air balloon systems demonstrate the feasibility of quantum communication in the condition of rapid motion, vibration, and random movement of satellites. At present, many countries including America, Canada, the European Union, China and Japan pay great attention to and support for accelerating the development of satellite-to-ground quantum communication. The first quantum satellite was launched in August 2016 in China, and will open a platform for satellite-to-ground quantum communication at an intercontinental level \cite{bib:N_535_478}.

In addition to the ongoing expansion in distance, QKD is also being developed for P2P communication with quantum networks, which may be multi-user and of various and diverse topological structures. There is much competition and cooperation in this area. The network of the American Defence Advanced Research Projects Agency (DARPA) connected the three nodes -- Harvard University, Boston University, and the BBN company -- in 2005, later increasing this to 10 nodes \cite{bib:QCC_2006_83}.

Since 2006, the EU has established a `SECOQ' network, combining the efforts of 41 research and industrial organisations from 12 countries, including the UK, France, Germany, and Austria.

A typical network employing a trusted repeater paradigm, with 6 nodes and 8 links was demonstrated in Vienna in 2008 \cite{bib:NJP_11_075001}. In 2010, the National Institute of Communication Technology, together with Nippon Telegraph \& Telephone Corporation (NTT), Nippon Electric Company, Mitsubishi Electric Corporation, Toshiba European company, Switzerland IDQ Company and an Austrian team constructed a Tokyo QKD network in a metropolitan area, demonstrating the world's first secure TV conferencing over a distance of 45km \cite{bib:OExp_19_10387}. The maximum distance is 9km, and the P2P bit-rate can reach 65kHz using superconducting detectors over 45km. \comment{How is the max distance both 45km and 9km???}

In China, quantum networks are also developing rapidly. In 2009, \cite{bib:OpEx17_6540} designed and constructed a 3-node network with a chained architecture, which demonstrated a cryptographically secure real-time voice call. In the same year, \cite{bib:OpEx_18_27217} designed a metropolitan all-pass and intercity quantum communication network in field fibre \comment{What's field fibre???} for four nodes. Any two nodes can be connected in the network, QKD between arbitrary pairs of users.

In 2009, \cite{bib:PLA_372_3957} constructed a QKD network with wavelength division multiplexers, realising 4 and 5 nodes with a star topology \cite{bib:OL_35_2454}.

In 2012, a Chinese team constructed the largest metropolitan area quantum network in Hefei, linking 46 nodes to allow real-time voice communications, text messages and file transfers. A more than 2,000km quantum communication channel used by government bodies and banks under construction in Beijing and Shanghai will soon be fully operational. With the help of the new satellite, scientists will be able to test QKD, and other entanglement-based protocols, between the satellite and ground stations, and conduct secure quantum communications between Beijing and Xinjiang's Urumqi.

With the distance and network coverage of quantum communication gradually increasing, the security of QKD systems draws more and more attention. Since 2012, the MDI QKD protocol has attracted much concern, because of its safety and practicability. \cite{bib:PRL_111_130501} demonstrated the protocol in the laboratory over more than 80km of spooled fibre with time-bin encoding. They also tried outdoor experiments over 18.6km.

A Brazilian group \cite{bib:PRA_88_052303} demonstrated the protocol using a polarisation encoding scheme. However, these two demonstrations did not really distribute random key bits between two parties, and thus were not full MDI QKD demonstrations. Additionally, their system can be attacked by PNS or USD \comment{Define these acronyms!} sources and cannot generate secure key-bits in principle. A full demonstration of time-bin phase-encoding MDI QKD was reported in \cite{bib:PRL_111_130502} over a 50km fibre link. \cite{bib:PRL_112_190503} implemented polarisation-encoded MDI QKD with commercial off-the-shelf devices over 10km, with a secure key-rate of 0.0047Hz. Subsequently, the Chinese group continues to upgrade the performance of MDI QKD systems, making them viable over distances of up to 200km \cite{bib:PRL_113_190501} and 400km \cite{bib:arx160606821}.

%
% Entanglement Purification
%

\section{Entanglement purification} \index{Entanglement purification}

\comment{Section complete}

\dropcap{E}{ntanglement} purification has been very successful in optical systems. Table.~\ref{tab:ent_pur} lists some of the major developments.

\begin{table*}[htpb]
\caption{Developments in experimental entanglement purification.} \label{tab:ent_pur}
\begin{tabular}{|p{0.755\linewidth}|p{0.22\linewidth}|}
	\hline
	Summary & References \& years \\
	\hline \hline
	An experimentally viable purification scheme, requiring only PBSs and post-selection, was proposed and demonstrated. The scheme was demonstrated again, using mixed states with fidelity of $0.75$ ($0.8$), which they were able to purify to $0.92$ ($0.94$), a major improvement. & \cite{bib:Nature_410_1067, bib:Nature_423_417} \\
	\hline
	A Bell experiment was performed using purified states. A state initially failing a Bell test successfully passed the test following entanglement purification. Unfortunately, the theoretical efficiency of the purification scheme is only $1/4$. & \cite{bib:PRL_94_040504, bib:Nature_410_1067} \\
	\hline
\end{tabular}
\end{table*}

%
% State Preparation
%

\section{State preparation} \index{State preparation}

%
% Coherent States
%

\subsection{Coherent states} \index{Coherent state preparation}

\comment{Section complete}

Coherent states are well approximated by laser or maser light. Nowadays thousands of types of lasers are known with different power, temporal, spatial and spectral parameters, and the technology is already very mature and available commercially. Here we only introduce some of the basic concepts of lasers related to quantum networking.

A laser can be classified as operating in either a continuous or pulsed regime. Most laser diodes used in communication systems are continuous. But they can also be externally carved at some rate by modulators to create pulsed light. Usually, pulsed lasers are created by the technique of Q-switching or mode-locking.

Different applications require lasers with different output power. Typical output powers of single-mode laser diodes are some tens of milliwatts, up to at most a few hundred of mW. However, multiple transverse mode diode lasers can reach up to some tens of Watts, and can be used as pump sources for high-quality and high-power single-mode solid-state lasers. Such single mode diode-pumped solid-state lasers can be further mode-locked to output femtosecond pulses, reaching as far as an average power of tens of Watts. Pulsed lasers can also be characterised with the peak power of each pulse. The peak power of a pulsed laser is many orders of magnitude greater than its average power.

Many quantum information experiments based on fibre networks \cite{sun2016quantum} have been performed using single-longitudinal-mode lasers, such as distributed feedback (DFB) lasers. The DFB laser has a stable wavelength that is etched by a grating, and can only be tuned slightly with temperature. Thus they are widely used in optical communication applications, such as dense wavelength division multiplexing (DWDM), where it is desired to have a tuneable signal and extremely narrow line width. Recently, several QKD schemes \cite{choi2011quantum, wang2015experimental} have been designed and implemented using multi-longitudinal-mode Fabry-Perot (FP) lasers\index{Fabry-Perot lasers}, mainly because their costs are significantly lower than DFB lasers.

%
% Single-Photon States
%

\subsection{Single-photon states} \index{Single-photon state preparation}

\comment{Section complete. But add table.}

A highly attenuated laser can be used as a good approximation to a single-photon source when no more than one single-photon is used in an interferometry experiment, such as QKD. Otherwise, heralded or deterministic single-photon sources are required.

The photons are usually created in pairs via SPDC, one photon (the heralding photon) can be used to herald the creation of another photon (the heralded photon). Recently, a state of the art SPDC source at a wavelength of 788nm was reported simultaneously with a high brightness of $\sim$12MHz/W, a collection efficiency of $\sim 70\%$ and an indistinguishability of $\sim 91\%$ \cite{bib:tenPhotEnt}. Although collection efficiencies can reach 90\% with a quasi-phase-matching technique, its brightness was usually limited to a lower level \cite{giustina2013, christensen2013}. Besides, heralded sources can be integrated via four-wave mixing (FWM) in waveguides  \cite{silverstone2014, spring2016} or optical fibres  \cite{goldschmidt2008, smith2009}. Despite being a probabilistic process, SPDC or FWM is technically mature and very (relatively) cheap, thus enjoying great popularity in quantum optics labs. That is why so far almost all the quantum information experiments in quantum networks are based on SPDC or FWM sources.

However, SPDC or FWM might not be easily scaled to arbitrary size due to higher-order emissions, or multiplexing requirements \cite{bib:RohdeLoopMulti15}. Thus, truly deterministic single-photon sources will be indispensable for future large-scale implementations. One of the most promising single-photon sources that has been developed is based on quantum dots, which simultaneously exhibit a high purity of 99.1\%, high indistinguishability of 98.5\%, and high extraction efficiency of 66\% \cite{he2013on, wei2014de, ding2016on, somaschi2016, wang2016near, loredo2016}. An excellent review of solid-state single-photon emitters was present by \cite{aharonovich2016solid}. \cite{eisaman2011} provides a comparison of different sources.

%
% Entangled States Based on Non-Linear Optics
%

\subsection{Entangled states based on non-linear optics} \index{Non-linear optics}

Multi-photon GHZ states based on SPDC \cite{kwiat1995new} can be step-by-step constructed with two-photon entangled states. Here we take the four photon one as an example \cite{pan2012multiphoton}. Assume that two SPDC sources emit two polarisation entangled states \mbox{$\frac{1}{2}(\ket{HH} + \ket{VV})^{\otimes 2}$}. After passing through the polarizing beam-splitter (PBS), only the superposition \mbox{$\frac{1}{\sqrt{2}}(\ket{HHHH} + \ket{VVVV})$}, which is a four-photon GHZ state, leads to fourfold coincidence. In 1999, the world's first multi-particle 3-photon entanglement was generated \cite{bouwmeester1999observation, pan2000experimental}. After that, Pan \textit{et al.} broke the records continuously, and realised 4- \cite{zhao2003experimental}, 5- \cite{zhao2004experimental}, 6- \cite{lu2007experimental}, 8- \cite{yao2012observation}, 10-photon entanglement \cite{bib:tenPhotEnt}, and 10-qubit hyper-entanglement with two degrees of freedom \cite{gao2010experimental}.

Optical cluster states have been realised with 4- \cite{walther2005experimental}, 5- \cite{lu2008experimental}, 6- \cite{lu2007experimental}, 8-photon \cite{yao2012experimental}, and 2-photon 4-qubit \cite{chen2007experimental} and 6-qubit \cite{ceccarelli2009experimental}.

N00N or Holland-Burnett states have been realised with photon-numbers from 2 \cite{edamatsu2002measurement} to 3 \cite{mitchell2004super}, 4 \cite{walther2004broglie, nagata2007beating, matthews2011heralding}, 5 \cite{afek2010high} and 6 \cite{xiang2012optimal} at visible wavelengths, and also at telecom wavelengths \cite{yabuno2012four, bisht2015spectral, jin2016detection}.

Continuous-variable entangled states are attractive because they exploit standard optical modulation and measurement equipment \cite{ralph2009bright}. One mode squeezed vacuum state can be produced by a degenerate optical parametric amplifier. For cat states \mbox{$\propto(\ket{\alpha} - \ket{-\alpha})$} with small `size' \mbox{$|\alpha|^2 \lesssim 1$}, it can be generated by reflecting a small fraction of squeezed vacuum state toward an avalanche photodiode (APD) \cite{neergaard2006generation, ourjoumtsev2006generating, wakui2007photon}. The APD will herald the subtraction of one photon, thus project the transmitted beam into the desired state. For large `size' cat state (e.g., \mbox{$|\alpha|^2 > 2.3$}), it can be prepared by reflecting a half of photon-number state $\ket{n}$ toward a momentum quadrature homodyne detector\index{Homodyne detection} \cite{ourjoumtsev2007generation,takahashi2008generation}. If the measurement outcome is close to 0, the transmitted mode is successfully projected into the desired state. Besides, two partite entangled cat states have been prepared and characterised \cite{ourjoumtsev2009preparation}. Initial demonstrations of the distillation of Gaussian entanglement have also been made \cite{takahashi2010entanglement, xiang2010heralded}.

%
% Non-Optical Systems
%

\section{Non-optical systems} \index{Matter qubits}

%
% Atomic Ensembles
%

\subsection{Atomic ensembles} \index{Atomic ensembles}

A variety of techniques have been developed to create squeezing and entanglement in atomic ensembles. The main methods exploit atom-light interactions in cold gases, or interaction between particles such as atom-atom collisions in Bose-Einstein condensates, or combined electrostatic and ion-light interaction in ion chains. Atom-light interaction currently represents the most mature method, and owns the highest squeezing of 20.1dB via an optical-cavity-based measurement \cite{hosten2016measurement}. Non-Gaussian states have also been produced recently in atomic ensembles. For instance, the detection of a single photon prepares almost 3000 atoms to an entangled Dicke state\index{Dicke state} \cite{mcconnell2015entanglement}. Different atomic ensembles can be entangled. For example, two ensembles can be entangled by storage of two entangled light fields \cite{lukin2000entanglement}, four quantum memories can be entangled via spin wave \cite{choi2010entanglement}, large ensembles of up to $10^4$ pair-correlated atoms in Bose-Einstein condensates can be entangled to a twin-Fock state \cite{lucke2011twin}. More contents can be found in some excellent review papers \cite{kimble2008quantum, hammerer2010quantum, sangouard2011quantum, pezze2016non}.

%
% Single Atoms
%

\subsection{Single atoms} \index{Single atoms}

Many experimental methods have been developed for measuring and manipulating individual quantum systems, including single atoms in a cavity, trapped ions, and neutral atoms in an optical lattice. In the atom-cavity system \cite{haroche2006exploring}, the largest Schr{\"o}dinger cat state was created with spin of 25 on a Rydberg atom \cite{facon2016sensitive}. In 2000, Haroche \textit{et al.} generated entanglement among two atoms and a single-photon cavity mode \cite{rauschenbeutel2000step}. Later, entanglement between two photons sequentially emitted by the same single atom in a cavity was demonstrated \cite{wilk2007single}. The entanglement fidelity of 86\% between the two photons was obtained, and importantly, the entanglement-generation efficiency is 15\%, a drastic improvement compared to free-space experiments with single atoms \cite{blinov2004observation}. Individual ions can be trapped in a special electric and/or magnetic fields \cite{leibfried2003quantum}. Multiple trapped ions can be entangled by a laser-induced coupling of the spins \cite{blatt2008entangled}. In the recent year, Blatt \textit{et al.} reported scalable and deterministic generation of multi-ions entanglement, including 8-qubit W state \cite{haffner2005scalable} and 14-qubit GHZ state \cite{monz2011}. An array of cold neutral atoms may be confined in free space by a pattern of crossed laser beams, forming an optical lattice. It is up to 50 qubits for the record of arbitrary coherent addressing of individual neutral atoms in a 3D optical lattice \cite{wang2015coherent}. For 2-qubit interaction, several groups reported recently generation and detection of two neural-atom entanglement with single-site-resolved technique \cite{kaufman2015entangling, islam2015measuring, dai2016generation}.

%
% Quantum Dots
%

\subsection{Quantum dots} \index{Quantum dot sources}

In quantum dots, the recombination of a pair of electrons and hole, named exciton, emits a single photon. There are also multiple electrons and/or holes leading to other transitions \cite{lodahl2015interfacing}. The simplest example is charged excitons, also called trions. Spontaneous emission from the trion state prepares a single electron spin in the quantum dot, while spin-photon entanglement was generated from such decay \cite{de2012quantum, gao2012observation}. Another example is a two electron-hole pairs, also called biexciton. Two photon polarisation-entangled states can be generated from biexciton-exciton cascade radiative decay in a single quantum dot \cite{muller2014demand}. The entangled photon pairs simultaneously show high purity of \mbox{$g_2(0) = 0.004$}, high fidelity of 0.81, high two-photon interference visibilities of 0.86 and on-demand generation efficiency of 0.86. Besides, a family of pyramidal site-controlled quantum dots allow areas with up to 15\% of polarisation-entangled photon emitters per chip to be obtained, with fidelities as high as 0.72 \cite{juska2013towards, mohan2010polarization}. Most recently, a cluster state of entangled five sequential photons was generated by periodic timed excitation of a matter qubit \cite{schwartz2016deterministic}. In each period, an entangled photon is added to the cluster state formed by the matter qubit and the previously emitted photons. More details about interfacing single photons and single quantum dots with photonic nano-structures can be found in several review papers \cite{de2013ultrafast, urbaszek2013nuclear, lodahl2015interfacing}.

%
% Nitrogren-Vacancy Centres
%

\subsection{Nitrogen-vacancy centres} \index{Nitrogen-vacancy (NV) centres}

An nitrogen-vacancy (NV) centres in diamond refers to a nitrogen (N) atom replacing a carbon atom and neighbouring one vacancy (V) \cite{doherty2013nitrogen}. In such centres, both electron and nuclear spins can exhibit long coherence times ($>$1ms for the electron spin and $>$1s for the nuclear spin) even at room temperature \cite{balasubramanian2009ultralong, neumann2010quantum, maurer2012room}. In 2008, Wrachtrup \textit{et al.} demonstrated the creation of bipartite- and tripartite-entangled states among single nuclear spins in diamond \cite{neumann2008multipartite}.

Five years later, they generated room-temperature entanglement between two single electron spins over some 10nm distance in diamond \cite{dolde2013room}. At the same year, Hanson \textit{et al.} reported heralded entanglement between two solid-state qubits located in independent low-temperature setups separated by 3m \cite{bernien2013heralded}.

Because NV centres couple to both optical and microwave fields, they can also be used as a quantum interface between optical and solid-state systems. For example, quantum entanglement between a polarised optical photon and a NV centre qubit has been realised in experiment \cite{togan2010quantum}.

%
% Superconducting Rings
%

\subsection{Superconducting rings} \index{Superconducting rings}

In superconductors, electrons are paired and condensed into a single macroscopic quantum state and construction of large quantum integrated circuits is promising \cite{devoret2013superconducting}. There are three basic types of superconducting qubits: charge, flux and phase. In 2007, a charge-insensitive qubit, named transmon, was designed \cite{koch2007charge}. Transmon qubit in a waveguide cavity, named circuit QED, has reached a coherence time on the order of 0.1 ms \cite{paik2011observation, rigetti2012superconducting}. In 2013, Martinis \textit{et al.} designed a cross-shaped transmon qubits, named Xmons, which balances coherence, connectivity and fast control \cite{barends2013coherent}. In the next year, they constructed a 5-qubit GHZ state using five Xmons arranged in a linear array \cite{barends2014superconducting}. Most recently, they reported the protection of states from environmental bit-flip errors in a nine Xmons qubits linear array \cite{kelly2015state}. Besides, the 100-photon Schr{\"o}dinger cat state was created mapping from an arbitrary transmon qubit state, and extended to superposition of up to four coherent states \cite{vlastakis2013deterministically}. The lifetime of microwave photon was extended to 0.3ms with error correction in superconducting circuits \cite{ofek2016extending}. Other topics about superconducting circuits interacting with other quantum systems can be found in an excellent review paper \cite{xiang2013hybrid}.

%
% Measurement
%

\section{Measurement} \index{Measurement}

%
% Photo-Detection
%

\subsection{Photo-detection} \index{Photo-detection}

There are a variety of single-photon detectors, detailed description of which can be found in several excellent review papers \cite{eisaman2011, hadfield2009}. Here we only introduce single-photon detectors that have good spectral response in the near-infrared region.

\textit{Single-photon avalanche diodes} (SPAD) -- They are the most compact and common single-photon detectors in the world. SPADs are avalanche diodes operated in Geiger mode (above breakdown voltage) to obtain a high gain. There are several commercially SPADs, with detection efficiency around 60\% at 780nm, maximum count-rate of 25MHz and dark-count-rate as low as 25Hz. Several remarks are listed as following:

\begin{itemize}
    \item Even higher detection efficiency is available. However, on account of the significance trade-off between detection efficiency, dark noise and dead-time, it is often at the expense of a lower maximum count-rate (typical dead-time is 1$\mu$s, leading to 1MHz count-rate) and a bigger dark noise.

    \item Telecom band SAPDs are usually based on InGaAs/InP, which, however, suffer from low efficiency about 25\% and long dead-time on the order of microsecond. Another more efficient method is up-conversion detection. The idea is to convert the telecom photons into the near infrared ones by up-conversion, and then detect them with a silicon SPAD. In this method, up-conversion usually utilises sum-frequency generation in periodically poled lithium niobate (PPLN) waveguides or bulk crystals. The detection efficiency can be as high as \mbox{$30\sim 40\%$} for PPLN waveguide-based up-conversion detectors \cite{shentu2013ultralow}.
    \item SPADs are also commercially available in a multi-channel (e.g., 4-channel) array format with single power supply and individual inputs and outputs. They are different from the multi-element SAPD arrays, which have either only one input or one output, and can be used to achieving photon number resolution or high-speed single-photon sensors.
\end{itemize}

\textit{Superconducting nanowire single photon detectors (SNSPD)} -- They are more efficient, but larger and more expensive than SPADs. Up until now, the highest detection efficiency of SNSPDs is 93\% at 1550nm, demonstrated in 2014 \cite{marsili2013}. SNSPDs are also commercially available. There are companies specialising in best-performance SNSPDs with high detection efficiency. They have achieved specifications with peak detection efficiencies higher than 80\% around 800nm, maximum dark-count-rates of \mbox{$100\sim 300$Hz}, and a dead-time ranging from \mbox{$10\sim 70$ns}. Similar performance is expected for optimised SNSPD at other wavelengths (\mbox{$780\sim 1550$nm}). However, SNSPDs must be operated at cryogenic temperatures of a few Kelvin to maintain the superconducting state in the sensing material. This is the main reason why SNSPDs are much more cumbersome and expensive than SPADs.

\textit{Transition edge sensors (TES)} -- They own the highest detection efficiency, even 98\% (95\%) at wavelengths around 850nm (1556nm) have been reported \cite{fukuda2011, lita2008}. Moreover, they are photon-number-resolving, owing to the linear response between resistance and temperature. However, the typical thermal recovery times of TESs are a fraction of a microsecond, limiting their application in high-speed quantum network. Besides, TESs usually work in a ultra-low temperature of 100mK, which makes it challenging to migrate the technique from lab to market.

%
% Homodyning
%

\subsection{Homodyning} \index{Homodyne detection}

The speed of many quantum information protocols (e.g., Continuous-variable QKD) is fundamentally limited by the bandwidth of the homodyne detector. The first time domain homodyne detection was performed below 1kHz and achieved a shot-noise to electronic-noise ratio of 9dB \cite{winzer2010}. So far several groups \cite{zavatta2002time, okubo2008pulse, kumar2012versatile, chi2011balanced, duan2013} have constructed a high speed and pulse-resolved homodyne detector in the near infrared and telecom wavelength regions, respectively. The state of the art of homodyne detection achieves a shot-noise to electronic-noise ratio of \mbox{$7.5\sim 14$dB}, and a bandwidth of \mbox{$100\sim 300$MHz}, which allows for from tens of MHz to one hundred of MHz repetition rate.

%
% Evolution of Optical States
%

\section{Evolution of optical states} \index{Evolution of optical states} \label{sec:LO_evolution}

%
% Optical Waveguides
%

\subsection{Optical waveguides} \index{Optical waveguides}

They can be integrated onto a small chip, which is free of alignment, space-saving and phase-stable.

\begin{itemize}
    \item Silica-on-silicon planar light-wave circuit (PLC) \cite{hibino2003silica}. For the state-of-the-art technique, PLCs can achieve ultra-low propagation loss ($<$0.01dB/cm) and coupling loss ($<$0.1dB/facet). Nowadays, small-scale quantum circuits have reached coupling loss $\sim$0.4dB/facet and each directional coupler (DC) loss $\sim$0.1dB \cite{carolan2015universal}. PLCs are micrometer waveguides due to its low index contrast $\sim$1.5\%. Usually, the bending radius is larger than 2 mm, allowing for DC length reached several millimetres at least \cite{carolan2015universal}. Therefore, PLC is thought as unacceptable for large scale circuit containing hundreds of DCs.

    \item Fused silica waveguides written by femtosecond laser direct written (FLDW) technology. Unlike PLC fabrication, the FLDW is a powerful and flexible technique for 3D rapid fabrication, requiring no masking procedure or cleanroom environment. FLDW waveguides exhibit lowest losses on the order of 0.1dB/cm \cite{sakuma2003ultra}. Besides, the maximum refractive index contrast achieved so far is only 1\%, which makes such waveguide larger than PLCs.

    \item Telecom silicon waveguides. They have higher refractive index, thus allowing a dramatic reduction in the size. For example, a \mbox{$2\times 2$} MMI coupler in silicon (27$\mu$m) \cite{bonneau2012quantum} is 40 times shorter than the one in silica (1.1mm) \cite{peruzzo2011}. The best propagation losses can reach 0.1dB/cm \cite{lee2000, gnan2008} and coupling losses of 0.5dB/facet is possible by using spot-size converters (SSC) \cite{almeida2003, mcnab2003}.
\end{itemize}

%
% Fibre-Loops
%

\subsection{Fibre-loops} \index{Time-bin encoding} \index{Fibre-loops}

It is extremely resource-frugal, irrespective of the size of the desired interferometer, whose scale is limited only by the loss rates of the fibre, dynamic switches and couplers \cite{motes2014}. While state-of-the-art integrated switches/couplers are fast enough, on the order of GHz \cite{winzer2010, schindler2014}, they involve high loss ($>1$dB). In order to lower loss \cite{he2016}, the controlled switch was realised using a bulk electro-optic modulator (EOM) with a transmission ratio of 97.3\%, modulated at 80MHz. The single-photons were coupled in and out of the loop using an acousto-optic modulator (AOM) with a transmission ratio of 99\% and first order diffraction efficiency of $\sim 85\%$. Inside the loop, the coupling efficiency from free-space to single-mode fibre was 92\%. Thus, a single-loop (BS operation) efficiency of 83\% was achieved.

%
% Others
%

\subsection{Others}

Single mode fibres can reach a ultra-low loss of 0.2dB/km at 1550nm. Bulk optical elements can be antireflection-coated to have a ultra-low reflection loss ($<0.01\%$).

%
% Quantum Memory
%

\section{Quantum memory} \index{Quantum memory}

\dropcap{M}{any} review articles can give a much more detailed account such as \cite{lvovsky2009optical, simon2010quantum, sangouard2011quantum, bussieres2013prospective, reiserer2015cavity}.

The simplest approach to storing light is an optical delay line, such as an optical fibre. This approach has been used to synchronise photons with the occurrence of certain events \cite{landry2007quantum}. However, the storage half-time is limited to tens of microseconds due to loss \cite{lvovsky2009optical}. Furthermore, the storage time of an optical delay is fixed by the delay length and on-demand output is impossible.

Alternatively, light can be stored in a high-Q cavity. The light cycles back and forth between the reflecting boundaries, allowing it to be injected into and retrieved from the cavity \cite{pittman2002single, pittman2002cyclical, leung2006quantum, maitre1997quantum, tanabe2007trapping, tanabe2009dynamic}. Unfortunately, the storage of light in cavities suffers from a tradeoff between short cycle time and long storage time, which limits the efficiency. Therefore, whereas optical delay lines and nano-cavities could be appropriate for obtaining on-demand single photons from heralded sources \cite{saglamyurek2015quantum, jin2015telecom}, they may not be suitable for quantum memory or quantum repeaters.

Quantum memories, related to quantum repeaters, can be mainly classified into two types. One is based on the Raman scattering, emitting one idler photon first and the stored photon later. Most recently, a quantum memory has been realised with simultaneously a readout efficiency of 76\% and a lifetime of 0.22s, which supports a sub-Hz entanglement distribution of up to 1,000km, for the first time going beyond the maximally achievable quantum communication distance using direct transmission. Another is based on such as electromagnetically induced transparency (EIT) or the atomic frequency comb (AFC), storing an arbitrary single-photon state and releasing it later.

Experimentalists strived either to push further the quantumness of the memories, or to improve their figures of merit. On the way towards the greatest quantumness, EIT in cold atomic ensembles was used to store entangled photon pairs \cite{Choi2008mapping} as well as squeezed vacuum pulses \cite{appel2008quantum, honda2008storage}. More recently, it was used to store a quantum bit encoded in the polarisation of a single photon with memory times hitting the millisecond range \cite{lettner2011remote, riedl2012bose, xu2013long}. On the way towards better and better figures of merit, recent experiments performed on classical signals have been reported with storage times up to the regime of one minute \cite{heinze2013stopped} or with efficiencies reaching 78\% \cite{chen2013coherent}. The (spatially) multimode nature of these storage media have also been probed experimentally \cite{ding2013single}.

There are also other promising approaches to quantum memories, e.g., quantum dots or crystalline defect centres. We note that the encoding of quantum information in the spin of individual nuclei in an otherwise spin-free or spin-polarised crystal lattice can be achieved with remarkably long coherence times \cite{steger2012quantum, saeedi2013room}, with a current record of several hours for the storage of strong light pulses in a spin ensemble \cite{zhong2015optically}. The major remaining challenge is to efficiently couple individual spins to photonic quantum channels. Typically, this requires exquisite control over the involved spatiotemporal light mode, posing a formidable challenge to system design and nano-fabrication \cite{reiserer2015cavity}.

%
% Quantum Computation
%

\section{Quantum computation}\index{Quantum computing}

\comment{What are the different physical architectures for QC? What are their records for the number of qubits they can implement? What kinds of fidelities do they exhibit?}

\subsection{Linear optics}\index{Universal linear optics quantum computation}

\subsection{Boson-sampling \& quantum walks}\index{Boson-sampling}\index{Quantum walks}

\subsection{Ion traps}\index{Ion traps}

\subsection{Superconducting rings}\index{Superconducting rings}

\subsection{Nitrogen-vacancy centres}\index{Nitrogen-vacancy (NV) centres}

\subsection{Adiabatic quantum computation \& quantum annealing}\index{Adiabatic quantum computation}\index{Quantum annealing}

\comment{What else? What other main architectures?}

\startalgtable