%
% The Quantum Ecosystem
%

\section{The quantum ecosystem}\index{Quantum ecosystem}\index{Ecosystems}

\dropcap{A}{ssociated} with any new computer platform comes a hardware/software \textit{ecosystem}\index{Ecosystems} that evolves around it.

If we consider the release of the original iPhone\index{iPhone} and its iOS\index{iOS} operating system, it wasn't just the product itself that was revolutionary, but the software that emerged surrounding it, and it wasn't until this software ecosystem emerged on the App Store\index{App Store} that the product realised its full potential and became truly transformative.

From the hardware perspective, it wasn't until interfacing standards such as USB\index{USB} emerged, allowing the plethora of competing hardware products to arbitrarily interconnect and interface with one another, that the hardware realised its full potential.

In the quantum era we anticipate the same phenomena to arise. What will this quantum ecosystem look like?

\begin{itemize}
\item Oracles: As an essential quantum software building block, oracles will become a fundamental unit for outsourcing. These oracles will store hard-corded or algorithmically-generated databases. For example, for use in genetic medicine (Sec.~\ref{sec:genetic_medicine}), such databases could algorithmically generate tables of candidate drug compounds, or they could implement mathematical functions whose input space is to be searched over when quantum-enhancing the solving of \textbf{NP}-complete problems.
\item Interfacing: \textit{De facto} standards will emerge for interconnecting quantum hardware units. Most notably, standards for optical interconnects will arise.
\item Modularisation: Arbitrarily-interconnectable units will develop, allowing quantum hardware to be constructed in an ad hoc, Lego-like\index{Lego} manner. These modules could implement small elements of a larger quantum computation, such as housing a small part of a larger graph state (Sec.~\ref{sec:module}), communications building blocks (such as transmitters or receivers of Bell-pairs), or algorithmic building blocks such as quantum Fourier transforms (Sec.~\ref{sec:QFT_alg}).
\end{itemize}

\comment{To do}