\section{Noise-Intermediate-Scale Quantum technology (NISQ)}

With the fault-tolerant universal quantum computers still being a long way away, we are now
at the early stages of exploring a new frontier of physics, which Preskill terms the 
``complexity frontier" or `` entanglement frontier" \cite{}.

\color{blue}
Things to talk about:
\begin{itemize}
\item  noisy
\item special purpose
\end{itemize}

\color{black}


We have good reasons to believe that a quantum computer can efficiently simulate any process that occurs in nature, we could probe deeply into the properties of complex molecules and materials. The confidence is based largely on 1) quantum complexity, and 2) quantum error correction. Both of these are based on entanglement, a correlation property amongst systems which are uniquely quantum mechanical.  We have strong evidence that quantum computers have abilities which surpass classical ones, three are as follows
\begin{itemize}
\item Quantum complexity - the best example is the Shor algorithm \cite{} which can find the factors of large numbers exponentially fast. Whilst we do not have a proof that a classical algorithm does not exist, mathematicians have been trying for decades, without success. Number factorization has significant impacts on cryptography because its computational complexity underpins all of the modern encryption. 
% 
\item Complexity theory arguments - computer scientists have shown that quantum states which can be easily prepared with a quantum computer have super-classical properties. Boson-sampling.
% 
\item No known classical algorithm can simulate a quantum computer.
\end{itemize}

A challenge is to understand which problems are hard for a classical computer but easy for a quantum one.


The huge obstacle that lies between us and building a quantum computer lies in the fact that we need to keep the system isolated from the outside, at the same time being able to control the system nearly perfectly, from the outside. Eventually, we expect to be able to protect quantum systems using quantum error correction. However, to perform quantum error correction, we expect to need $10^3 - 10^4$ physical qubits to encode each logical qubit. This makes a huge difference in the number of qubits we need to control individually. Therefore, reliable fault-tolerant computers with quantum error correction are not likely to be available in the near future.



NISQ refers to the size of quantum processors which may be available in the next few years, with around 50 to a few hundred qubits. The number 50 is a significant milestone because that is what we can simulate by brute force with the most powerful existing computer \cite{}. The main question is when will quantum computes be able to solve useful problems faster than the classical computer?
Here we discuss a few potential uses of NISQ.


\subsection{Quantum optimizers}
for many problems, there is a big gap between the approximation achieved by classical algorithms and the barrier of NP-hardness. Quantum computers are not expected to solve the worst-case NP-hard problems, however, quantum devices may be able to find better approximate solutions to such problems, or find such approximations faster. The vision for using NISQ to solve optimization problems is a hybrid quantum-classical algorithm. In this scheme we use the quantum device to produce a $n-$qubit state, measure the qubits, then process the measurement outcomes classically; then this is used as a feedback in the next quantum state preparation. The cycle is repeated until it converges to a quantum state from which the approximation can be extracted. Two such algorithms go by the name quantum approximate optimization algorithm\cite{}, and variational quantum eigensolver \cite{}.


\subsection{
Quantum machine learning}
Much of the literature build on quantum algorithms which speed up linear algebra
One of the potentials of quantum machine learning rests upon QRAM -- quantum random access memory. For classical data processing, by using QRAM we may be able to represent a large amount of classical data, N-bits, using $\log$N qubits. However, the bottleneck may be in the encoding/decoding of the QRAM, which may eliminate potential advantages. 
It is perhaps more natural to think about quantum machine learning in a setting where both the input an output are quantum states, e.g. to control a quantum system, or in learning probability distributions where entanglement plays an important role.

\subsection{Quantum semidefinite programming}

\subsection{Quantum simulation}
As it has been stressed previously, quantum computers are very well suited for studying highly entangled systems of many particles. It is the natural platform to simulate highly entangled states, which is where quantum computers appear to have a clear advantage over classical ones.
 
With a universal quantum computer, we anticipate that quantum chemistry will be facilitated by quantum computing. This can be used to design farmaceuticals, as well as new catalysts which can improve the efficiency of nitrogen fixation or carbon capture. We may be able to find new materials that can lead to more efficient electricity transmission. However, these promises may not be fulfilled with NISQ, because algorithms to accurately simulate large molecules and materials may not succeed without quantum error correction.

We may be able to study quantum dynamics using NISQ: one example is quantum chaos, systems in which entanglement spreads very rapidly. Classical computers are particularly inefficient at simulating quantum dynamics, and here quantum computers have an especially obvious advantage. Insights can be gained using noisy devices with orders of 100's of qubits.

