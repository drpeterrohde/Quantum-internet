%
% Quality of Service
%

\section{Quality of service (QoS)}\index{Quantum error correction (QEC)}\index{Quality of service}

\comment{To do}

As discussed previously in Sec.~\ref{sec:QOS}\index{Quality of service}, quality of service is a major consideration in any quantum network. If we are transmitting quantum states over a quantum channel, there will inevitably be deterioration in the form of decoherence, loss, and other undesirable effects we wish to mitigate. In this section we review some of the essential quantum error correction (QEC) techniques that can be used to achieve this.

\comment{Latin quote}\index{Latin}

%
% Entanglement Purification
%

\subsection{Entanglement purification}\index{Entanglement purification}

\comment{To do - move from protocols section. Here do circuit description. Leave optical implementation in protocols section.}

%
% 3-Qubit Code
%

\subsection{3-qubit code}\index{3-qubit code}

\comment{To do}
\comment{Syndrome extraction}\index{Syndrome extraction}

%
% 9-Qubit Code
%

\subsection{9-qubit code}\index{9-qubit code}

\comment{To do}

%
% Stabiliser Codes
%

\subsection{Stabiliser codes}\index{Stabiliser codes}

\comment{To do}

%
% Surface Codes
%

\subsection{Surface codes}\index{Surface codes}

\comment{These are a particular example of stabiliser codes with an elegant graphical interpretation}

\comment{Missing degree of freedom in stabilisers defines qubit. How does topology (genus) relate to number of logical qubits?}

If the intention is to perform quantum computations using cluster states shared over a network, QEC and fault-tolerance \textit{must} be taken into consideration, or catastrophic algorithmic failure will inevitably follow -- the cluster state model is no different from the circuit model in this respect. Fault-tolerance theory places hard thresholds on the amount of noise (typically depolarising errors and loss) qubits may be subject to in order for fault-tolerance to be possible and computations to succeed. This places strict QoS (Sec.~\ref{sec:QOS}) constraints on the network, which can be accommodated for using the usual depolarising and efficiency cost metrics when developing networking strategies and link performance requirements.

It has been shown that fault-tolerance is possible within the cluster state model \cite{bib:NielsenDawson04, bib:Dawson06} using variations of conventional QEC codes. However, more importantly, from cluster states certain \textit{topological QEC codes} \cite{???} can be readily constructed. This implements a form of QEC-encoded measurement-based quantum computing protocol, where the computation proceeds in a measurement-based fashion, but is `natively' fault-tolerant.

These codes have been shown to have very favourable fault-tolerance thresholds in terms of both depolarising noise and loss \cite{bib:StaceBarrettDohertyLoss, bib:BarrettStaceFT}, as well as frugal resource overhead compared to traditional concatenated codes. Additionally, loss- and gate-failure-tolerant codes, uniquely applicable to the cluster state model, have been described, with very favourable loss thresholds \cite{bib:Varnava05, bib:RalphHayes05, bib:Duan05}. 

Importantly, topological codes do not require joint measurements across the entire graph state, instead requiring only operations localised to small regions within the graph. Thanks to this, computation using such topological codes can remain distributed, without requiring the entire state to be held locally by a particular host, or requiring full access to the entire state by any particular user.

The most common topological code, which we will use here as an example, is the toric code\index{Toric code}, which resides on a lattice graph over the surface of a torus\footnote{As with cluster states, this graph needn't (but could) correspond to a network graph.}. As with cluster states (Sec.~\ref{sec:CSQC}), the toric code is most easily visualised in the stabiliser formalism\index{Stabiliser formalism}. Consider a rectangular sub-graph of the torus. We place a qubit on each edge (not vertex) of the graph. Now we define two sets of stabiliser operators: \textit{star} and \textit{plaquette} operators,
\begin{align} \index{Topological code stabilisers}\index{Star operator}\index{Plaquette operator}
	\hat{S}_\text{star}(v) &= \prod_{i\in e(v)} \hat{X}_i, \nonumber \\
	\hat{S}_\text{plaquette}(p) &= \prod_{i\in e(p)} \hat{Z}_i,
\end{align}
where $e(v)$ are the edges neighbouring vertex $v$, and $e(p)$ are the edges surrounding plaquette $p$. By definition, the toric code state, $\ket\psi_\text{toric}$, satisfies the stabiliser relations,
\begin{align}
	\hat{S}_\text{star}(v) \ket\psi_\text{toric} &= \ket\psi_\text{toric} \,\forall\, v, \nonumber \\
	\hat{S}_\text{plaquette}(p) \ket\psi_\text{toric} &= \ket\psi_\text{toric} \,\forall\, p.
\end{align}
Unlike the cluster state stabilisers from Eq.~(\ref{eq:CS_stab}), these stabilisers are insufficient to fully characterise a unique quantum state. Rather, there are two unspecified degrees of freedom, which allows for a single qubit to be represented. Modifications of the topology, in the form of holes in the lattice (the genus of the topology), allow larger numbers of qubits to be encoded. Logical operations are implemented by performing gates and measurements across topologies over the surface.

The important feature to note is that logical qubits encoded into the toric code do not reside locally at any of the physical qubits in the topology. Rather, they reside jointly across the entire graph, which, like cluster states, might be partitioned across multiple hosts, enabling distributed computation.

This is all summarised in Fig.~\ref{fig:toric_code}.

\begin{figure}[!htb]
	\includegraphics[width=0.47\textwidth]{toric_code}
	\caption{Graph representation of the toric QEC code, and its associated stabilisers. The star and plaquette stabilisers across all vertices, jointly specify the state of the graph up to two missing degrees of freedom, which encode a single logical qubit. Thus, a logical qubit is encoded jointly across the entire graph, not at any specific vertex. Logical operations are performed via operations following topological paths through the lattice (not shown). The graph may be distributed across multiple hosts for distributed quantum computation.} \label{fig:toric_code}
\end{figure}

\comment{How to convert cluster states to topological codes. Add discussion of how to perform gates.}

Having defined the toric code as such, QEC proceeds in a similar manner to any other stabiliser code -- we measure all the stabilisers, yielding a syndrome, from which we can determine geometrically where errors took place in the graph, which can subsequently be corrected (if below threshold).

The simplest example of error detection is the scenario where a single bit-flip ($\hat{X})$ error has occurred in the graph. Now exactly two plaquette stabilisers will yield the $-1$ measurement outcome, instead of the expected $+1$ outcome. These two stabilisers will necessarily be neighbouring ones, overlapping at the qubit where the error took place. Thus, using this geometric property, we are able to identify the location of the single $\hat{X}$ error and subsequently correct it. On the other hand, if there were too many errors, it is possible they could conspire against us to create ambiguity in the geometric argument for the location of the errors. \comment{Figure for both examples of this -- under and over threshold!}

Importantly, the stabilisers are all defined over geometrically localised neighbourhood regions, and do not require long-range measurements, making this type of code suitable to distributed models for quantum computation.

\comment{What about the actual computation? Can this still be distributed when we do the topological gates etc?}

\comment{Figures for thresholds FT and QEC}

%
%
% Topological Codes
%

\subsection{Topological codes} \label{sec:topol_codes} \index{Topological codes}

\comment{Toric code}

%
% Unitary Error Averaging
%

\subsection{Unitary error averaging} \index{Quantum error correction}\index{Unitary error averaging}\label{sec:error_averaging}

\comment{Redundant copies could in principle be distributed}

\comment{To do}

\comment{Add figure}

%
% Gate Failure Codes
%

\subsection{Gate failure codes}\index{Gate failure codes}

\comment{To do}

\comment{Cluster states}

%
% Qubit Loss Codes
%

\subsection{Qubit loss codes}\index{Qubit loss codes}

\comment{cite RohdeHaselgrove}

\comment{Located vs unlocated errors}

\comment{Tree horticultural scheme Rudolph et al.}

%
% Decoherence-Free Subspaces
%

\subsection{Decoherence-free subspaces}\index{Decoherence-free subspaces}

The QoS techniques discussed previously were based on the notion of performing measurements on quantum systems to project them into subspaces devoid of errors. For example, in the 3-qubit code, measurement of the syndrome qubits projects the encoded state into a subspace where there was either no error, or in which an error occurred whose location is known and may therefore be corrected.

An alternate approach is to encode quantum information into Hilbert spaces which are invariant under a given error model. Such spaces are referred to as \textit{decoherence-free subspaces} (DFSs). In this instance we assume the error model is known, for example a dephasing channel, such that we can choose the appropriate DFS.

To illustrate this idea we will consider encoding a single logical qubit into two physical qubits. The error model we will encode against is a collective $Z$-rotation, where the two physical qubits are subject to perfectly correlated $Z$ errors. This arises naturally in the context of, say, atomic qubits subject to the same external electromagnetic field, and therefore accumulate the associated phase errors in tandem.

A single-qubit $Z$-rotation of angle $\theta$ on the Bloch sphere\index{Bloch sphere} is given by,
\begin{align}
	\hat{Z}(\theta) &= e^{i\frac{\theta}{2}\hat{Z}}\nonumber \\
	&= \left(\begin{matrix}
  e^{i\frac{\theta}{2}} & 1 \\
  0 & e^{-i\frac{\theta}{2}}
\end{matrix}\right).
\end{align}
where $\hat{Z}$ is the usual Pauli phase-flip operator\index{Pauli operators},
\begin{align}
\hat{Z}=\ket{0}\bra{0}-\ket{1}\bra{1}.	
\end{align}
This operates on the physical basis states as,
\begin{align}
	\hat{Z}(\theta) \ket{0} &\to e^{i\frac{\theta}{2}}\ket{0}, \nonumber \\
	\hat{Z}(\theta) \ket{1} &\to e^{-i\frac{\theta}{2}}\ket{1}.
\end{align}

Now we employ the encoding for logical basis states,
\begin{align}
\ket{0}_L &\equiv \ket{0}_1\otimes\ket{1}_2,\nonumber \\
\ket{1}_L &\equiv \ket{1}_1\otimes\ket{0}_2,
\end{align}
Note that both logical basis states are invariant under a common $Z$-rotation (the other two physical basis states, $\ket{0}_1\otimes\ket{0}_2$ and $\ket{1}_1\otimes\ket{1}_2$ do not observe this property),
\begin{align}
	\hat{Z}_1(\theta)\hat{Z}_2(\theta)\ket{0}_L &= \ket{0}_L,\nonumber\\
	\hat{Z}_1(\theta)\hat{Z}_2(\theta)\ket{1}_L &= \ket{1}_L.
\end{align}
Thus, when acting on an arbitrary linear combination of these basis states (i.e a logical qubit),
\begin{align}
\ket\psi_L = \alpha \ket{0}_L + \beta\ket{1}_L,
\end{align}
via linearity the logical qubit must also be invariant under the collective error,
\begin{align}
	\hat{Z}_1(\theta)\hat{Z}_2(\theta)\ket\psi_L = \ket\psi_L.
\end{align}
This type of DFS encoding therefore protects a logical qubit against arbitrary correlated $Z$-rotations.

The same principle can be logically extended to many other correlated error models. For example, operating in a rotated basis (under a Hadamard transform), one could similarly protect against correlated $X$-rotations using the encoding,
\begin{align}
\ket{0}_L &\equiv \ket{-}\otimes\ket{+},\nonumber \\
\ket{1}_L &\equiv \ket{+}\otimes\ket{-},
\end{align}
where \mbox{$\ket\pm=\frac{1}{\sqrt{2}}(\ket{0}\pm\ket{1})$}.

This idea, although very simple, is very powerful, since it is a completely passive form of error correction, requiring no syndrome measurements, feedforward or correction operations. It also arises quite naturally in some systems where external fields act roughly uniformly across the physical qubits within a system.

%
% Dynamical Decoupling
%

\subsection{Dynamical decoupling}\index{Dynamical decoupling}

An alternate mechanism by which errors could be introduced into our system is via coupling to an external environment (for example via an electromagnetic field) introducing a persistent evolution of our qubits, which is slow-moving compared to the rate at which the implemented computation is evolving the system. We can model this as a joint system/environment Hamiltonian of the form,
\begin{align}\label{eq:dyn_dec_ham}
\hat{H}_\mathrm{total} = \lambda_\mathrm{comp}\hat{H}_\mathrm{comp} + \lambda_\mathrm{env}\hat{H}_\mathrm{env} + \lambda_\mathrm{int}\hat{H}_\mathrm{int},	
\end{align}
where the different components represent, in order, the Hamiltonians of the: total joint system; quantum computer (or system of interest); environment; interaction between system and environment. We are specifically operating in the regime where \mbox{$\lambda_\mathrm{int}\ll\lambda_\mathrm{comp}$}, such that the computation is the dominant term in the evolution and the environmental coupling can be treated as a small perturbation from the desired evolution.

The goal of dynamical decoupling is to minimise the influence of the system/environment interaction term, $\hat{H}_\mathrm{int}$, by manipulating the system in such a way that this term continuously cancels itself out over time.

Let us illustrate how this can be achieved using a simple example, whereby a single-qubit system couples with an environment which introduces a slow, unknown phase evolution. Discretising time, we can write this phase evolution as a Pauli $Z$-rotation on the Bloch sphere\index{Bloch sphere},
\begin{align}
	\hat{Z}(\theta) &= e^{i\frac{\theta}{2}\hat{Z}}\nonumber \\
	&= \left(\begin{matrix}
  e^{i\frac{\theta}{2}} & 1 \\
  0 & e^{-i\frac{\theta}{2}}
\end{matrix}\right),
\end{align}
for some unknown, but small $\theta$. Next we observe that, quite obviously, a $Z$-rotation of angle $\theta$ can be trivially undone by applying another $Z$-rotation of angle $-\theta$, since,
\begin{align}
\hat{Z}(-\theta)\hat{Z}(\theta) = \hat{I}.	
\end{align}

Therefore, if the system/environment coupling introduced an evolution of $\hat{Z}(\theta)$ in the previous unit of time, our goal is to manipulate it into implementing $\hat{Z}(-\theta)$ during the next one. Alas, the environment is beyond our control and we cannot directly order it to reverse direction. We do, however, have complete control over our qubit system, so we will achieve the same outcome by flipping the direction of the Bloch sphere underneath the environments foot, allow it to take a step forward, before flipping it back.

How do we achieve this flip in the Bloch sphere? Simply by using the following identity from the algebra of the Pauli matrices\index{Pauli operators},
\begin{align}
\hat{X}\hat{Z}(\theta)\hat{X} = 	\hat{Z}(-\theta).
\end{align}
That is, applying a bit-flip to a qubit, followed by an arbitrary phase-rotation, followed by another bit-flip is equivalent to having taken the same phase-rotation in the reverse direction. Effectively we are tricking the environment into time-reversal!

Now if we proceed for two time-steps, once bit-flipped and another not, we have,
\begin{align}
\hat{Z}(\theta)\cdot[\hat{X}\hat{Z}(\theta)\hat{X}] = \hat{I},
\end{align}
and the unknown phase-rotation has been eliminated.

Proceeding over the course of a lengthy evolution, we will implement the dynamical decoupling by continuously applying bit-flips to our qubit, such that in conclusion there are no accumulated unknown phase-evolutions. The control sequence for implementing this is sometimes referred to as `bang-bang' control\index{Bang-bang control}, since we are repeatedly implementing bit-flips at a fast rate.

As described above, we have merely error corrected a quantum memory. Of course we wish to implement far more sophisticated evolutions. This requires breaking down the computational evolution (given by $\hat{H}_\mathrm{comp}$) into a large number of small, discrete steps. These are interspersed with our bang-bang control sequence so as to continuously remove any phase-errors accumulated during the course of the computation.

Because the evolution implemented by the computation is constantly changing the state of our computational qubit, the change in trajectory introduced by the dephasing error model is also varying over time. But for the phase cancellation to work, the forward and backward erroneous steps ($\hat{Z}(\pm\theta)$) must be along the same path -- i.e they must all couple-up. This mandates that the bang-bang bit-flip control sequence be executed at a rate very fast compared to the rate of change in the phase-error -- if we fail to implement them fast enough, erroneous steps in the forward and backward directions will not pair up exactly, leaving some residual accumulated errors.

Dynamical decoupling extends to all manner of error models, beyond the simple $Z$-error model presented above. They are therefore a very powerful tool in error correction. However, unfortunately they are only naturally suited to continuous evolutions governed by Hamiltonians of the form shown in (Eq.~\ref{eq:dyn_dec_ham}), not to the more common discretised models such as the circuit and cluster state models.

\comment{Reference, Bloch sphere figures}

%
% Continuous-Variable Quantum Error Correction
%

\subsection{Continuous-variable quantum error correction}\index{Continuous-variable quantum error correction}

\comment{To do}