%
% Optical Routers
%

\section{Optical routers} \index{Optical!Routers}

\dropcap{P}{erhaps} the most fundamental building block in any network is routers, devices which switch data packets between multiple inputs and outputs so as to relay them to a destination. Indeed, in many real-world networks, many nodes will purely implement routing, and nothing more elaborate such as computations or other end-user protocols, to be discussed in Part.~\ref{part:protocols}.

We now discuss the implementation of optical routers, beginning with the simplest two-port switch, upon which we build to construct more general and powerful routers.

There are many parameters of interest characterising the operation of optical routes. We will introduce the terminology convention that:
\begin{itemize}
	\item \textit{Ports}\index{Ports}: number of input and output optical modes in a device.
	\item \textit{Channels}\index{Channels}: number of simultaneous communications streams running in parallel through the device.
	\item \textit{Optical depth}\index{Optical!Depth}: number of primitive optical elements/devices an optical path traverses through the course of its trajectory from input to output.
	\item \textit{Directionality}\index{Directionality}: whether information is transferred in one (unidirectional) or two (bidirectional) directions.
	\item \textit{Switching time}\index{Switching!Time}: time for the switch to be reconfigured from one state to another.
	\item \textit{Delay time}\index{Delay time}: time taken by signal to reach the output line of the switch from input.
	\item \textit{Throughput}\index{Throughput}: maximum data rate that can flow through the switch.
	\item \textit{Switching energy}\index{Switching!Energy}: energy input required for activating and deactivating the switch.
	\item \textit{Power dissipation}\index{Power!Dissipation}: power dissipated during the process of switching.
	\item \textit{Insertion loss}\index{Insertion loss}: loss in signal power when the switch is connected.
	\item \textit{Crosstalk}\index{Crosstalk}: coupling to other optical modes.
\end{itemize}

A summary of the routing devices we consider, and their associated resource requirements, is provided in Tab.~\ref{tab:router_summary}.

Of course, real-world routers will not only switch optical paths, but also implement some (probably undesired) quantum processes across those paths, such as a loss channel or temporal mode-mismatch. Thus, proper analysis of optical router performance in quantum networks requires treating them as legitimate nodes in the network graph, with associated costs and attributes, as per the QTCP framework.

\startnormtable
\begin{table*}[!htbp]
	\begin{tabular}{|c|c|c|}
		\hline
  		Device & Resource requirements & Optical depth \\
  		\hline
  		\hline
  		Two-channel two-port switch & \mbox{$N_\mathrm{bs}=2$}, \mbox{$N_\mathrm{ps}=1$} & \mbox{$d=1$} \\
  		Linear $n$-port multiplexer & \mbox{$N_\mathrm{s}=n-1$} & \mbox{$1\leq d\leq n-1$} \\
  		Pyramid $n$-port multiplexer & \mbox{$N_\mathrm{s}=n-1$} & \mbox{$d=\log_2 n$} \\
    	Single-channel multi-port switch (linear) & \mbox{$N_\mathrm{s}=2n-3$} & \mbox{$2\leq d\leq 2n-3$} \\
  		Single-channel multi-port switch (pyramid) & \mbox{$N_\mathrm{s}=2n-3$} & \mbox{$d=2\,\log_2 n-1$} \\
  		Multi-channel multi-port switch & \mbox{$N_\mathrm{s} = \left\lceil \frac{n^2}{2}\right\rceil - n + 1$} & \mbox{$\left\lceil \frac{n}{2} \right\rceil \leq d\leq n-1$} \\
  		Crossbar switch & \mbox{$N_\mathrm{s}=n^2$} & \mbox{$1\leq d\leq 2n-1$}\\
    	\hline
	\end{tabular}
	\captionspacetab \caption{Summary of different primitives for constructing optical routers. $n$, $N_\mathrm{bs}$, $N_\mathrm{ps}$ and $N_\mathrm{s}$ are the number of input/output ports, beamsplitters, phase-shifters, and two-port switches respectively. $d$ is the optical depth (in units of number of two-port switches). Since all of the multi-port devices are constructed from two-port switches, in all cases \mbox{$N_\mathrm{bs} = 2 N_\mathrm{s}$} and \mbox{$N_\mathrm{ps} = N_\mathrm{s}$}.} \label{tab:router_summary} \index{Optical!Routers}\index{Optical!Depth}\index{Optical!Routers!Resource requirements}
\end{table*}
\startalgtable

%
% Mechanical Switches
%

\subsection{Mechanical switches}\index{Mechanical switches}

Most obviously, optical switching could be performed mechanically, by physically displacing fibre endpoints, directing them towards different routes\footnote{Remember, the telephone network used to be mechanically routed by human switchboard operators, manually routing point-to-point connections!}. Such switches have found use in other areas, but are not particularly appropriate for quantum information processing applications, as they are extremely slow compared to electro- or acousto-optic technologies. Certainly, mechanical switching would not be applicable to optical fast-feedforward, such as that required by optical quantum computing, on the order of nanoseconds.

A second disadvantage of mechanical switches is that the introduction of moving parts into quantum optics protocols makes optical stabilisation extremely challenging. The mechanical control required to preserve wavelength-level coherence, for example, is effectively ruled out by moving mechanical parts.

%
% Interferometric Routers
%

\subsection{Interferometric switches} \index{Interferometric!Switches}\label{sec:interfer_switches}

\sectionby{Rohit Ramakrishnan}

Interferometric routers are based on the principle that the evolution implemented by interferometers are in general highly dependent on the phase relationships within them. This reduces the seemingly uphill task of high-speed, dynamic switching between modes to the problem of implementing dynamically-controllable phases. Thankfully there are a number of techniques for implementing such phase-switching. We will discuss these phase-modulation techniques, before moving onto combining them into more complex routing systems.

A phase modulator\index{Phase!Modulators} is a classically-controlled device that lets us tune the local phase accumulated by an optical path, ideally over the full range of $\{0,2\pi\}$. These may be implemented in several ways:

%
% Electro-Optic Modulators
%

\subsubsection{Electro-optic modulators} \index{Electro-optic modulators}

Electro-optics modulators (EOMs) are based on anisotropic materials\index{Anisotropic materials}, in which the refractive index\index{Refractive index} changes according to an applied electric field. There are two primary variations on this:
\begin{itemize}
	\item Pockel's effect\index{Pockel's effect}: a linear electro-optic effect, where the refractive index\index{Refractive index} change is proportional to the applied electric field.
	\item Kerr's effect\index{Kerr's!Effect}: a quadratic electro-optic effect, where the refractive index change is proportional to the square of the applied electric field.
\end{itemize}

These changes in refractive index are typically small, such that the effects are significant over propagation distances larger than the light's wavelength. For example, in a material where the refractive index increases by $10^{-4}$, an optical wave propagating a distance of $10^{-4}$ wavelengths will acquire a phase-shift of $2\pi$.

The refractive index of an electro-optic medium\index{Electro-optic medium} is a function $n(E)$ of the applied electric field $E$. This function varies only slightly with $E$, such that using a Taylor series expansion\index{Taylor series} about \mbox{$E=0$} we obtain,
\begin{align}
n(E) = n+a_1E + \frac{1}{2}a_2E^2+\dots.
\end{align}

In a Pockel's medium\index{Pockel's medium} this relation becomes (after approximating and simplifying),
\begin{align}
n(E) = n-\frac{1}{2}\chi n^3 E,
\end{align}
where $\chi$ is called the Pockel's coefficient\index{Pockel's coefficient} or linear electro-optic coefficient. Typical values of $\chi$ lie in the range $10^{-12}-10^{-10}$mV$^{-1}$. The most common crystals used as the medium for Pockel's cells are NH$_4$H$_2$PO$_4$ (ADP), KH$_2$PO$_4$ (KDP), LiNbO$_3$, LiTaO$_3$, and CdTe.

In a centrosymmetric material or Kerr's medium\index{Kerr's!Medium} this relation becomes (again after approximating and simplifying),
\begin{align}
n(E) = n-\frac{1}{2}\xi n^3 E^2,
\end{align}
where $\xi$ is the Kerr's coefficient\index{Kerr's!Coefficient} or the quadratic electro-optic coefficient. Typical values of $\xi$ lies in the range $10^{-18}-10^{-14}$m$^2$V$^{-2}$.

The refractive index profiles as a function of applied electric field strength for Kerr's and Pockel's mediums are shown in Fig.~\ref{fig:EOM_ref_index}.

\begin{figure}[!htbp]
\includegraphics[clip=true, width=0.475\textwidth]{pockels_medium} \if 2\pubmode \\ \fi
\includegraphics[clip=true, width=0.475\textwidth]{kerrs_medium}
\captionspacefig \caption{Dependence of refractive index on electric field in: Pockel's medium\index{Pockel's medium}, exhibiting linear electric field dependence; Kerr's medium\index{Kerr's!Medium}, exhibiting quadratic electric field dependence. Graphs express qualitative behaviour only, hence no numbers are provided.}\label{fig:EOM_ref_index}
\end{figure}

Light transmitted through a transparent plate with controllable refractive index undergoes a controllable phase-shift. This plate can be used as an optical phase modulator.

Consider light traversing a Pockets cell of length $L$ to which an electric field $E$ is applied. The phase-shift undergone is given by,
\begin{align}
\phi \approx \phi_0 - \pi\frac{\chi n^3 E L}{\lambda_0},
\end{align}
where,
\begin{align}
\phi_0 = \frac{2\pi nL}{\lambda_0}.
\end{align}

If the electric field generated by applying a voltage $V$ across the faces of the cell of dimension $d$ is,
\begin{align}
	E=\frac{V}{d},
\end{align}
then,
\begin{align}
	\phi=\phi_0-\pi \frac{V}{V_\pi},
\end{align}
where,
\begin{align}
	V_\pi=\frac{d\lambda_0}{L\chi n^3},
\end{align}
is the half-wave voltage\index{Half-wave voltage}, the voltage at which the phase-shift changes by $\pi$.

The electric field is applied either perpendicular (transverse modulators\index{Transverse modulators}) or parallel (longitudinal modulators\index{Longitudinal modulators}) to the direction of the propagation light. The value of the electro-optic coefficient $\chi$ depends on the directions of propagation and the applied field. The speed of operation is limited by the capacitive effects and the transit time of the signal through the material. 

State of the art electro-optic modulators are integrated optic devices based on LiNbO$_3$, in which materials like titanium are used to increase the refractive index. The typical operation speed is above 100GHz. Light signals can be coupled in and out using optical fibres.

%
% Acousto-Optic Modulators
%

\subsubsection{Acousto-optic modulators} \index{Acousto-optic modulators}

Sound, or acoustic waves\index{Acoustic waves}, are vibrations that travel through a medium with a velocity characteristic of the medium. This can create perturbations in the refractive index of the optical medium, thus modifying the velocity of light passing through the medium. Thus sound can be used to modify the effect of the medium on light. That is, sound can control the direction of propagation of light. This acousto-optic effect is used to make a variety of devices like optical modulators, switches, deflectors\index{Deflectors}, filters\index{Filters}, isolators\index{Isolators}, frequency shifters\index{Frequency!Shifters} and spectrum analysers\index{Spectrum analysers}. This is shown in Fig.~\ref{fig:AOM}.

\begin{figure}[!htbp]
\includegraphics[clip=true, width=0.35\textwidth]{AOM}
\captionspacefig \caption{Acousto-optic modulators as a classically-controlled optical switch. The light signal is refracted depending on the applied sound wave.}\label{fig:AOM}
\end{figure}

According to quantum theory, a light wave of angular frequency $\omega$ and wave-vector $k$ is a stream of photons each with energy $\hbar\omega$ and angular momentum $\hbar k$. Additionally, acoustic waves with frequency $\Omega$ and wave-vector $q$ are a stream of phonons each with energy $\hbar\Omega$ and momentum $\hbar q$. When light and sound interact, a photon combines with a phonon to generate a new photon with energy and wave-vector subject to energy and momentum conservation laws\index{Conservation!Energy}\index{Conservation!Momentum}, 
\begin{align}\label{eq:AOM_energy}
\hbar\omega_r &= \hbar\omega + \hbar\Omega,\nonumber\\
\hbar k_r &= \hbar k + \hbar q.
\end{align}
The associated energy conservation diagram is shown in Fig.~\ref{fig:AOM_energy_diagram}.

\begin{figure}[!htbp]
\includegraphics[clip=true, width=0.25\textwidth]{AOM_energy_diagram}
\captionspacefig \caption{Energy diagram for an acousto-optic modulator, based on Eq.~(\ref{eq:AOM_energy}). Energy and momentum must be conserved from the incident photon of energy $\hbar\omega$ and phonon of energy $\hbar\Omega$, yielding a scattered photon of energy $\hbar\omega_r$.}\label{fig:AOM_energy_diagram}
\end{figure}

Since the intensity of the reflected light is proportional to the intensity of the sound (provided the intensity of sound is low), the intensity of reflected light can be varied proportionally by using an electrically controlled acoustic transducer. This device can be used as a linear modulator of light.

When the acoustic power increases beyond a certain threshold level, total reflection of light occurs whereby the modulator behaves as an optical switch. By switching the sound on and off, the reflected light can be turned on and off, yielding an acoustically-controlled switch.

%
% Magneto-Optic Modulators
%

\subsubsection{Magneto-optic modulators} \index{Magneto-optic modulators}

In the presence of a static magnetic field, certain materials act as polarisation rotators, known as the Faraday effect\index{Faraday effect}. The angle of rotation is proportional to distance and the rotary power\index{Rotary power} $\rho$ (angle per unit length), which is proportional to the component $B$ of the magnetic flux density in the direction of wave propagation,
\begin{align}
	\rho=VB,
\end{align}
where $V$ is known as the Verdet constant\index{Verdet constant}, which is a function of wavelength $\lambda_0$.
 
Examples of materials that exhibit the Faraday effect include glass, Yttrium-iron-garnet (YIG), Terbium-gallium-garnet (TGG) and Terbium-aluminium-garnet (TbAlG).

A simple form of magneto-optic modulator comprises a parallel-sided disk of material placed in a small coil. An alternating current in the coil provides a magnetic field normal to the plane of the disk. The material becomes magnetised in this direction and light propagating through the disk undergoes a polarisation rotation about its plane of polarisation. The modulation of the angle of the plane of polarisation induced by the alternating current may be converted to amplitude modulation\index{Amplitude modulation} by subsequently passing the beam through a polariser. 

%
% Two-Channel Two-Port Switches
%

\subsection{Two-channel two-port switches} \index{Two-channel two-port switches}

The elementary primitive switch from which more complicated routers may be constructed is the two-channel two-port switch. This switch may be constructed from a Mach-Zehnder interferometer\index{Mach-Zehnder (MZ) interference}, with a classically-controlled phase-shifter in one arm. By switching the phase to either \mbox{$\phi=0$} or \mbox{$\phi=\pi$}, the MZ may be tuned to implement either an identity or swap operation respectively. This is shown in Fig.~\ref{fig:two_channel_two_port_switch}.

In the upcoming diagrams we present, arrows are used to indicate the time-ordering of the flow of data. However, it should be noted that a MZ interferometer is reversible and therefore bidirectional, and so too are all of the more complex routers based upon them.

\begin{figure}[!htbp]
\includegraphics[clip=true, width=0.475\textwidth]{two_channel_two_port_switch}
\captionspacefig \caption{(top) A two-channel two-port switch has two inputs and two outputs, implementing either an identity or swap operation between them. This may be constructed using a Mach-Zehnder interferometer with a variable, classically-controlled phase-shift, $e^{i\phi}$, in one of the arms, which may be implemented using an acousto-optic or electro-optic modulator (AOM or EOM). The phase-shift is allowed to be either \mbox{$\phi=0$} for an identity channel (bottom left) or \mbox{$\phi=\pi$} for a swap operation (bottom right). Because the switch is based on MZ interference, this technique only applies to optical states which undergo MZ interference. The total resource requirements are two 50:50 beamsplitters and a single phase-shifter.} \label{fig:two_channel_two_port_switch} \index{Two-channel two-port switches}\index{Mach-Zehnder (MZ) interference}
\end{figure}

Because the two-port switch is based upon MZ interference, it will only function for optical states subject to such MZ interference. Thus, single-photons and coherent states are applicable, whereas thermal states, for example, are not.

%
% Multiplexers & Demultiplexers
%

\subsection{Multiplexers \& demultiplexers} \index{Multiplexers}\index{Demultiplexers}

From the two-port switch, which implements a controlled permutation of two optical modes, we can construct multi-port multiplexers and demultiplexers, which controllably route a single input port to one of $n$ multiple output ports, or vice versa.

There are two main architectures that may be employed for implementing such multiplexers/demultiplexers. The first is to use a linear cascade of two-port switches, shown in Fig.~\ref{fig:linear_multiplexer}\index{Linear multiplexers \& demultiplexers}. The second is to use a pyramid cascade, shown in Fig.~\ref{fig:pyramid_multiplexer}\index{Pyramid multiplexers \& demultiplexers}. Both layouts require,
\begin{align}
N_\mathrm{s} = n-1,
\end{align}
two-port switches to implement. However, they differ in one important respect. In the linear multiplexer, different routes experience different optical depth\index{Optical!Depth}, ranging from \mbox{$d=1$} (for the first port) to \mbox{$d=n-1$} (for the final port). This will lead to asymmetry in accumulated errors. In the pyramid multiplexer, on the other hand, all optical paths have the same optical depth, \mbox{$d=\log_2 n$}, yielding completely symmetric operation.

The differing optical depths of linear and pyramid multiplexers lend themselves naturally to different applications. Suppose that in a network a single input-to-output route through a multiplexer is used far more often than the others. In that case, utilising a linear multiplexer will minimise average optical depth since that route can be designated to the first output port, which has an optical depth of only \mbox{$d=1$}. On the other hand, in a very balanced network, in which all optical routes are used roughly uniformly, the average case logarithmic optical depth of the pyramid multiplexer outperforms the average case linear optical depth of the linear multiplexer.

Note that the logarithmic optical depth of the pyramid configuration grows less quickly than the linear average optical depth of the linear configuration. Thus, on average, optical paths pass through fewer optical elements in the pyramid configuration, reducing average accumulated error rates when using noisy optical elements. This, in conjunction with the pyramid's perfect symmetry, makes the pyramid multiplexer configuration generally most favourable.

\begin{figure}[!htbp]
\includegraphics[clip=true, width=0.425\textwidth]{linear_multiplexer}
\captionspacefig \caption{Linear multiplexers (left) and demultiplexers (right) may be constructed from a linear chain of two-port switches (grey boxes), cascading into one another. These switch a single optical channel between $n$ ports. The total resource requirements are \mbox{$n-1$} two-port switches. The optical depth ranges from $1$ (for the first port) to \mbox{$n-1$} (for the final port).} \label{fig:linear_multiplexer} \index{Linear multiplexers \& demultiplexers}
\end{figure}

\begin{figure}[!htbp]
\includegraphics[clip=true, width=0.35\textwidth]{pyramid_multiplexer}
\captionspacefig \caption{Pyramid multiplexers (top) and demultiplexers (bottom) decompose the multiplexing into a binary tree-structure of two-port switches (grey boxes), shown here for the case of \mbox{$n=8$} ports. For $n$ ports, all optical paths observe an optical depth of \mbox{$d=\log_2(n)$} two-port switches, of which there are \mbox{$n-1$} in total.} \label{fig:pyramid_multiplexer} \index{Pyramid multiplexers \& demultiplexers}
\end{figure}

%
% Single-Channel Multi-Port Switches
%

\subsection{Single-channel multi-port switches} \index{Single-channel multi-port switches}

The multiplexers and demultiplexers route between one port and $n$ ports. In the more general and useful case, we wish to route between $n$ inputs and $n$ outputs. If we only require one active channel at a given time, such a router may be trivially constructed from an $n$-port multiplexer connected to and $n$-port demultiplexer, as shown in Fig.~\ref{fig:single_channel_multi_port_switch}. Here, the demultiplexer chooses one of the input modes to route to its single output, which then feeds into the multiplexer to fan it out to the desired output. The multiplexers/demultiplexers could be implemented using either of the aforementioned layouts, yielding a total resource count of,
\begin{align}
	N_\mathrm{s} = 2n-3,
\end{align}
two-port switches\footnote{Note that the multiplexer and demultiplexer each require \mbox{$2(n-1)$} two-port switches, but one of the central ones adjoining the multiplexer and demultiplexer is redundant and may be eliminated, reducing the number of two-port switches to \mbox{$2n-3$}.}.

\begin{figure}[!htbp]
\includegraphics[clip=true, width=0.375\textwidth]{single_channel_multi_port_switch}
\captionspacefig \caption{A single-channel multi-port switch may be constructed by demultiplexing the $n$ input ports to a single port, routing the desired input channel to that port, before multiplexing it back out to the desired output port. This allows an arbitrary input to be routed to an arbitrary output, but only one channel at a time. This requires \mbox{$2n-3$} two-port switches in total.} \label{fig:single_channel_multi_port_switch} \index{Single-channel multi-port switches}	
\end{figure}

%
% Multi-Channel Multi-Port Switches
%

\subsection{Multi-channel multi-port switches} \index{Multi-channel multi-port switches}

The single-channel multi-port switch enables switching between an arbitrary number of input/output ports, but suffers that it can only route a single channel at a time. The most general scenario to consider is multi-channel multi-port switching, which implements an arbitrary permutation between $n$ inputs and $n$ outputs. That is, all $n$ ports may be routing active channels, enabling simultaneous routing of multiple data-flows.

Such a switch may be constructed from a staggered, rectangular lattice of two-port switches, as shown in Fig.~\ref{fig:multi_channel_multi_port_switch}. It is easy to see upon inspection that optical paths exist between every input/output pair of ports. The total resource count for this device is,
\begin{align}
N_\mathrm{s} = \left\lceil \frac{n^2}{2}\right\rceil - n + 1,
\end{align}
two-port switches.

The operation implemented by this device can therefore be expressed as,
\begin{align}
	\begin{pmatrix}
  		\hat{b}^\dag_1 \\
  		\hat{b}^\dag_2 \\
  		\vdots \\
  		\hat{b}^\dag_m
\end{pmatrix}=\hat\sigma \cdot \begin{pmatrix}
  		\hat{a}^\dag_1 \\
  		\hat{a}^\dag_2 \\
  		\vdots \\
  		\hat{a}^\dag_m
\end{pmatrix},
\end{align}
where \mbox{$\hat\sigma\in S_m$} is an arbitrary element of the symmetric group (i.e a permutation matrix), and $\hat{a}_i^\dag$ ($\hat{b}_i^\dag$) are the input (output) photonic creation operators.

\begin{figure}[!htbp]
\includegraphics[clip=true, width=0.475\textwidth]{multi_channel_multi_port_switch}
\captionspacefig \caption{A completely general multi-channel multi-port switch may be constructed using a staggered grid of two-port switches (grey boxes), shown here for \mbox{$n=6$} ports. This allows the implementation of an arbitrary permutation between input and output ports, enabling all $n$ channels to be simultaneously utilised and routed across distinct input-to-output routes. This requires \mbox{$\left\lceil \frac{n^2}{2}\right\rceil - n + 1$} two-port switches in total. Optical depth is approximately equal across all input-to-output paths.} \label{fig:multi_channel_multi_port_switch} \index{Multi-channel multi-port switches}	
\end{figure}

Note that this decomposition is more favourable than the completely general Reck \textit{et al.} decomposition presented in Fig.~\ref{fig:LO_archs}(a), since the circuit is balanced, with (almost!) identical optical depths across all input-to-output paths.

%
% Crossbar Switches
%

\subsection{Crossbar switches}\index{Crossbar switches}

A general multi-port switching architecture, that gained popularity in the early days of channel-switched telecommunications networks\index{Channel-switched networks}, is the crossbar architecture, whereby $n$ inputs are mapped to $n$ outputs via a binary permutation matrix, which controls a lattice of \mbox{$2\times 2$} switches. The general layout of the architecture is shown in Fig.~\ref{fig:crossbar_switch}, and an example of a routing sequence corresponding to a particular binary control matrix is shown in Fig.~\ref{fig:crossbar_example}.

\begin{figure}[!htbp]
\includegraphics[clip=true, width=0.4\textwidth]{crossbar_switch}
\captionspacefig \caption{Crossbar architecture for multi-port switching. Each orange box represents a \mbox{$2\times 2$} switch, of any physical implementation. The switching sequence of the constituent two-port switches is defined by a binary \mbox{$n\times n$} permutation matrix, whose elements determine whether a given two-port switch flips modes or doesn't.} \label{fig:crossbar_switch}	
\end{figure}

\begin{figure}[!htbp]
\includegraphics[clip=true, width=0.4\textwidth]{crossbar_example}
\captionspacefig \caption{Example switching configuration for a \mbox{$5\times 5$} crossbar switch. The switch colours represent whether the respective \mbox{$2\times 2$} switch is set to flip (red) the modes or not (orange).} \label{fig:crossbar_example}	
\end{figure}

Clearly the scheme requires $n^2$ two-port switches to implement arbitrary \mbox{$n\times n$} mode permutations. The main disadvantage of this scheme is that in general different paths within a given permutation experience differing optical depths, ranging from 1 (best case) to \mbox{$2n-1$} (worst case).