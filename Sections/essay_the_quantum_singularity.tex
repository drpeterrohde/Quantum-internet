\subsection{The quantum singularity} \label{sec:singularity} \index{Quantum singularity}

Since there is positive computational leverage associated with quantum networking -- to all participating parties -- this creates a self-reinforcing economic quantum ecosystem, whereby all market participants are incentivised to continue contributing to the network, exponentially enhancing its collective power, with ever increasing returns over time. That is, the computational dividend increases exponentially over time, with every new qubit being more valuable (computationally) than the last.

Once we reach the point at which the economic returns of the global virtual quantum computer outweigh the costs of maintaining the quantum internet and the participating quantum computers, we will have reached the \textit{quantum singularity} -- a point of no return with self-reinforcing exponential growth, driven by market forces.

What are the implications of the quantum singularity? Foremost, investment into quantum hardware will provide guaranteed profit, since we are operating in a regime of monotonically increasing dividends, with positive dollar-value return on investment into hardware. \comment{Double check that this is consistent with pricing models}

Since quantum algorithms provide quadratic speedup to \textbf{NP}-complete problems, it follows that the existing distributed quantum computer will be able to perform self-enhancement by improving network routing efficiency compared to what could be achieved using classical scheduling algorithms. This self-improvement will also be self-reinforcing -- as the quantum network becomes more powerful, its self-improvement capability will accelerate, at which point the self-enhancement becomes self-sustaining, without the need for human intervention or classical computers to guide the way.

It is to be expected that the quantum internet will become to a large extent autonomous after the point of singularity. It will perform self-improvements via enhanced optimisation of itself. It will guarantee investment by the market into network growth, owing to computational dividends exceeding investment, with guaranteed growth (assuming the network does not contract in size over time).

\comment{To do! talk about machine learning. Relate to self-improvement and self-learning.}

\textit{--- Adiuva nos Deus.}

\comment{Complete this section}

\comment{Talk about self-improvement -- optimising network; routing; optimising architectural designs (check up on circuit layout algorithms, are they NP-complete?)}