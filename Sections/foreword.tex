%
% Foreword
%

\famousquote{Any sufficiently advanced technology is indistinguishable from magic.}{Arthur C. Clarke}
\newline \newline
\famousquote{There may be babblers, wholly ignorant of mathematics, who dare to condemn my hypothesis, upon the authority of some part of the Bible twisted to suit their purpose. I value them not, and scorn their unfounded judgment.}{Nicolaus Copernicus}
\newline \newline
\famousquote{Imagination is more important than knowledge.}{Albert Einstein}

\section{Foreword}\label{sec:foreword}

\dropcap{Q}{uantum} technologies\index{Quantum technologies} are not just of interest to quantum physicists, but will have transformative effects across countless areas -- the next technological revolution. For this reason, this work is directed at a general audience of not only quantum computer scientists, but also classical computer scientists, physicists, economists, artists, musicians, and computer, software and network engineers. More broadly, we hope this work will be of interest to those who recognise the future significance of quantum technologies, and the implications (or even just curiosities) that globally networking them might have -- the creation of the global quantum internet \cite{bib:Kimble2008}. We expect the answer to that question will look very different to what emerged from the classical internet.

A basic understanding of quantum mechanics \cite{bib:Sakurai94}, quantum optics \cite{bib:GerryKnight05}, quantum computing and quantum information theory \cite{bib:NielsenChuang00}\footnote{Throughout this manuscript we use the Nielsen \& Chuang convention for the pronunciation of `zed' \cite{bib:NielsenChuang00}.\index{Zed}}, and classical networking \cite{bib:TanenbaumNet} are helpful, but not essential, to following our discussion. Some mathematical sections require a basic understanding of the mathematical notation of quantum mechanics. Although the reader without this background ought to be able to nonetheless follow the broader arguments.

The entirely technically disinterested or mathematically incompetent reader may refer to just Secs.~\ref{sec:introduction}, \ref{sec:outlook} \& \ref{sec:vision_quant} -- essentially brief non-technical, highly speculative essays about the motivation, applications and implications of the future quantum internet.

This work is partially a review of existing knowledge relevant to quantum networking, and partially original ideas, to a large extent based on the adaptation of classical networking concepts and quantum information theory to the context of quantum networking. A reader with an existing background in these areas could calmly skip the respective review sections.

Our goal is to present a broadly accessible technical and non-technical overview of how we foresee quantum technologies to operate in the era of quantum globalisation, and the exciting possibilities and emergent phenomena that will evolve from it.

The theme music for this work may be found at \texttt{\href{http://soundcloud.com/peter-rohde/wir-sind-ein-volk}{http://tiny.cc/f5k5vy}} \copyright\index{Theme music}\footnote{We use the copyright (\copyright) and trademark (\texttrademark) symbols liberally throughout this text when referring to things where we believe commercial opportunities may exist in the future. The authors do not own copyrights or trademarks to anything in this text.}.