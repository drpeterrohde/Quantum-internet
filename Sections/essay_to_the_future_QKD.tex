\section{The future of QKD}
% 
Quantum key distribution (QKD) is the first quantum information task
to reach commercialization. Its purpose is to distribute a secret
key between two trusted parties who share a quantum channel,
as well as a classical channel for authentication. Unlike the current
cryptography system which is based on the limit of computational power of an adversary, the security of QKD only assume the laws of physics, and a 
correct implementation.


The typical setting of QKD is as follows. The two trusted parties who
want to establish a secret key are conventionally named Alice and Bob.
They share two channels, a quantum channel, which allows them to send
quantum state to one another, and a classical channel, with which they can
send classical messages. The communication over the classical channel is
assume to be public, and the eavesdropper (Eve) has free access to without 
ever being detected, however, Eve cannot modify the messages shared over
the classical channel.

 The quantum channel, is also subjected to possible manipulation from an
 eavesdropper. The task of Alice and Bob is thus to guarantee security
against an adversarial eavesdropper. The typical protocol assumes that
Alice and Bob do not share any secret to begin with.
The origin of the security of QKD springs from the fundamentals
of quantum mechanics, that is, any action by an observer on
a quantum state necessarily induces a change in the state. This means
that in combination with classical communication, actions of an eavesdropper cannot go undetected.


The ultimate goal of a QKD network is long distance secure quantum communication with imperfect sources.
% 
Despite the significant advances in both the theoretical and experimental development of QKD, a number of challenges remain for QKD to be widely used 
in securing daily communications\cite{RevModPhys.81.1301,diamanti2016practical}. Experimentally, much effort is being put into improving the performance of the QKD system. On the theory side, 
showing the security of a QKD system with finite key size is also a challenge,  
because information-theoretic security is achieved only when immunity against the most general (coherent) attack is proven \cite{diamanti2016practical}.


\section{Performance}
Some of the criteria for assessing the performance of a QKD scheme include key rate, distance, cost and robustness.
% 
\subsection{Key rate}
Currently, a strong disparity exists between classical and quantum key distribution rates. Classical optical communication delivers in the order of 100 Gbits per second per wavelength, whereas Mbit/s rates are achieved with QKD.
% 
 The obtained key rate depends on the performance of the detector used. For QKD based on single-photon detection techniques, to achieve a high bit rate, one requires true single-photon states, in combination with detectors with high efficiency and short dead time. The current developments are promising, with a reported quantum efficiency of $93\%$ at telecom wavelengths \cite{marsili2013detecting}.
% 
 For continuous-variable QKD systems, increasing the bandwidth of the homodyne/heterodyne detectors whilst keeping the electronic noise low is essential.

\subsection{Distance}
Extending the range of QKD systems is a major challenge and driving factor for QKD in terms of future network applications. Basically, two approaches are being pursued -- free space and repeaters.  A quantum repeater, similar
to its classical analogue, is a device that can extend the range of quantum communication between sender and receiver. However, one cannot amplify the signal that contains the quantum information, which we know based on the no-cloning theorem. The no-cloning theorem states that one cannot make copies of an unknown quantum state whilst leaving the original one unchanged, which is the basis of the security of QKD.

A quantum repeater needs to effectively restore the quantum information without measuring it directly, and is a component which is extremely technologically challenging. 
Over optical fibre networks, the standard loss for 1550 nm is 0.2 dB/km. Over enough distance, this unavoidable loss will reduce the key rate to a level of little practical relevance, therefore a ground-based solution would be to divide the entire channel into segments, where two partners exchange pairs of entangled photons and store it in a quantum memory\cite{PhysRevLett.81.5932,PhysRevA.59.169}.

The second is to use free space quantum communication techniques via satellite links. Satellite QKD is reachable with the present technology. Here satellites are used as intermediate trusted nodes for
communication between locations on the ground.
 Direct links can be
established between ground stations and the satellite, thus
enabling communication between parties separated by long distances. Satellite QKD suffers less loss, because the quantum channel is mostly free-space once $\approx$ 10 km above sea level.
% 
In China, satellite QKD over 1200 km has been demonstrated \cite{liao2017satellite}.


\subsection{Cost and robustness}
For QKD systems to be consumer-friendly, low cost and robustness are crucial features. Preferably QKD systems should make use of existing data fibre-optic infrastructure since the use of dark fibres are not only expensive, but often unavailable\cite{diamanti2016practical}. Single-photon detectors at room temperatures are also desirable, 
because this can remove the requirement for cryogenic cooling, hence reduces power consumption.

Integrated photonic platforms are being explored to reduce the cost, since miniaturization can lead to light-weight, low-cost QKD modules that can be mass-manufactured. 
% 4
Currently, two platforms are being explored, silicon\cite{lim2014review} and indium phosphide\cite{smit2014introduction}.
A reconfigurable QKD system which uses an In-P transmitter and silicon detectors has been demonstrated in the laboratory\cite{sibson2017chip}.




\section{New protocols}
In parallel to hardware development, research effort is directed at finding new QKD protocols which can out-perform the existing ones. Two of these are high-dimensional (HD) QKD and the Round-Robin differential phase shift protocol (RR-DPS).

High-dimensional QKD aims at encoding more than one bit in each detected photon, which can increase the information capacity when the photon rate is limited.
Security proof against collective attacks are being developed, and an experiment has demonstrated an information capacity
 6.9 bits per coincidence rate at 2.7 Mbit/s over 20 km \cite{zhong2015photon}.

The Round-Robin DPS protocol\cite{sasaki2014practical} removes the need to monitor signal disturbance. In a conventional QKD protocol, the noise parameter needs to
be estimated; and if high precision is required, the portion of the signal that is sacrificed increases, thus decreasing the efficiency of the protocol \cite{cai2009finite,hayashi2014security}. This protocol has a high tolerance to
the quantum-bit error rate ($< 50\%$)\cite{xu2015discrete}, and makes it attractive for implementation when high systematic errors are unavoidable.   

However, currently, neither of the protocols out-compete the more mature decoy-state BB84.

\section{Challenge in security}
Although QKD protocols can be proven to be secure from an information-theoretic perspective, physical implementations often contain imperfections which are not considered in the model. Attacks can be designed to exploit such imperfections, on either the source or the detectors.
% 
Table ~\ref{attacks} (From Ref.~\cite{lo2014secure}) summarises some attacks against certain commercial and research systems.

\begin{table}
\begin{tabular}{ |c|c|c| } 
 \hline
 Attack &  Targeted component & tested system\\ 
  \hline
Time shift\cite{qi2005time,PhysRevA.78.042333,PhysRevA.74.022313}
        & detector & commercial \\
        % 
Time information\cite{lamas2007breaking}  & detector & research \\
Detector control \cite{lydersen2010hacking,yuan2010avoiding}  & detector  &   commercial \\
Detector control \cite{gerhardt2011full}    & detector  & research  \\
Detector dead time\cite{weier2011quantum}     & detector  & research         \\
Channel calibration\cite{jain2011device}    & detector  &  commercial        \\
Phase remapping\cite{xu2010experimental} &  phase modulator & commercial \\
Phase information\cite{tang2013source} & Source & research           \\
Device calibration\cite{jouguet2013preventing} & local oscillator & research \\
                \hline
\end{tabular}
\caption{\label{attacks} Summary of various attacks against some commercial and 
research QKD systems}
\end{table}


To regain security, a number of solutions have been proposed. 

\subsection{QKD with imperfect sources}

The source is typically less vulnerable to attacks because Alice can prepare her quantum states in
a protected environment, and we expect that she can characterize her source. Therefore flaws in the state preparation can be incorporated into the security proof.


A loss-tolerant protocols has been proposed \cite{PhysRevA.90.052314}, and further developed by Ref\cite{PhysRevA.92.032305}, where
decoy-state QKD with tight finite-key security has been employed. A wide range of imperfections with the laser source has been taken into account in Ref.~\cite{mizutani2015finite}, including intensity fluctuation.  A security proof has shown that perfect phase randomization is also not necessary \cite{cao2015discrete}.


These are strong evidence which suggests that secure quantum communication with imperfect sources is feasible\cite{diamanti2016practical}. Intuitively, QKD with imperfect sources are feasible, because by assuming that the states prepared by Alice are qubits, then Eve cannot unambiguously discriminate Alice's states\cite{diamanti2016practical}. 



\subsection{MID-QKD}

To prove security for Bob's measurement device is more problematic, since Eve has complete access to the quantum channel and she can send in any signal. 

One candidate for a long-term solution to side-channel attacks is device-independent (DI) QKD \cite{PhysRevLett.98.230501}. It relies on the violation of a Bell inequality \cite{hensen2015loophole}, and the security can be proven without knowledge of the implementation. However, the expected secure key rate is low even at short distances. A more practical approach is measurement-device independence (MDI) \cite{PhysRevLett.108.130503}, which is immune to side-channel attacks against the measurement device. Here the device is treated as a black box, and does not need to be trusted. An important assumption for MDI QKD is that Eve cannot interfere with the state preparation process. 

Another candidate such as 
detector-device-independent\cite{lim2014detector,PhysRevA.92.022337} (DDI) QKD has been designed to take advantage of both the strong security of MDI-QKD, with the efficiency of conventional QKD. However, DDI-QKD has been shown to be vulnerable to attacks \cite{PhysRevLett.117.250505}. 

The MID-QKD protocol has been extended to the continuous variable (CV) framework. However, this system requires homodyne detectors with efficiency $> 85\%$, and a reliable phase reference between Alice and Bob.  


\section{Conclusion}

We have discussed some significant remaining challenges in QKD.
These range from theoretical security proofs to hardware developments.
Advances in QKD will not only enable point-to-point quantum communication,
but have implications for a range of network applications, such as
quantum secret-sharing\cite{cleve1999share,PhysRevA.61.042311,PhysRevA.71.044301}, blind quantum computing\cite{broadbent2009universal,barz2012demonstration}, quantum anonymous
broadcasting\cite{christandl2005quantum} and more.

As remarked in Ref.~\cite{diamanti2016practical},
``determining the exact power and limitations of quantum
communication is the subject of intense research efforts worldwide.
The formidable developments that can be expected in the
next few years will mark important milestones towards the
quantum internet of the future. "  

\bibliography{reference}

