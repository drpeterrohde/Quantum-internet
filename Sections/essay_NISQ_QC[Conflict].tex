\section{Noisy-Intermediate-Scale Quantum technology (NISQ)}
NISQ refers to the size of quantum processors which may be available in the next few years, with around 50 to a few hundred qubits. These are going to be noisy and will not have full quantum-error correcting capabilities. They are likely to be special-purpose devices \cite{preskill2018quantum}.

The fault-tolerant universal quantum computers still a long way away. Nevertheless, 
with advances in quantum control,
we are now at a position to explore a new frontier of physics, where we have entanglement
as part of our toolbox.



We have believe that a quantum computer may be able to efficiently simulate any process that occurs in nature, which will enable us to study the properties of complex molecules and new materials. 
Our confidence is based quantum complexity arguments, and our eventual capabilities to perform
 quantum error correction. 
Both of these are based on entanglement, a correlation property amongst systems which are uniquely quantum mechanical. We have strong evidence that quantum computers have abilities which are not
accessible to a classical, three pieces are as follows
\begin{itemize}
\item Quantum complexity - we think that what some tasks which are computationally difficult classically
may be easy quantumly. The best example is the Shor algorithm \cite{shor1994algorithms},
 which allows one to factorise large numbers exponentially faster. 
 Whilst we do not have a proof that a classical algorithm that can efficiently factorise does not exist, brilliant mathematicians have been trying for decades without success. Number factorisation has significant impacts on cryptography because its computational complexity underpins all of the modern encryption. 
% 
\item Complexity theory arguments - computer scientists have shown that quantum states which can be easily prepared with a quantum computer have super-classical properties. Given single photons are input
into an interferometer, it is hard for a classical computer to generate the probability distributions at the output. On the other hand, a quantum computer can simply perform the experiment.
% 
\item No known classical algorithm can simulate a quantum computer.
\end{itemize}

As we see, there is a clear distinction between what is hard classically and and quantumly. Intense research effort is now dedicated to understand which problems are hard for a classical computer but easy for a quantum one.


The huge obstacle that lies between us and building a quantum computer is that we need to keep the system isolated from the environment, at the same time being able to control it nearly perfectly. Eventually, we expect to be able to protect quantum systems using quantum error correction. 
However, in order to perform quantum error correction, we think we need $10^3 - 10^4$ physical qubits to encode each logical qubit. This adds a huge overhead to the number of qubits we need to control individually. Therefore, reliable fault-tolerant computers with quantum error correction are not likely going to be available in the near future.


In terms of the number of qubits , the number 50 is a significant because that is approximately the number of qubits we can still simulate by brute force with the most powerful existing computer \cite{boixo2018characterizing}. The main question is: when will quantum computes be able to solve useful problems faster than the classical computer?
Here we discuss a few potential uses of NISQ.


\subsection{Quantum optimizers}
For many problems, there is a big gap between the approximation achieved by classical algorithms and the barrier of NP-hardness. We do not expect quantum computers are not expected to solve the worst-case NP-hard problems, however, quantum devices may be able to find better approximate solutions to such problems, or find such approximations faster. The vision for using NISQ to solve optimization problems is a hybrid quantum-classical algorithm. In this scheme we use the quantum device to produce a $n-$qubit state, measure the qubits, then process the measurement outcomes classically; then this is used as a feedback in the next quantum state preparation. The cycle is repeated until it converges to a quantum state from which the approximation can be extracted. Two such algorithms go by the name quantum approximate optimization algorithm\cite{farhi2014quantum}, and variational quantum eigensolver \cite{mcclean2016theory}.


\subsection{Quantum machine learning}
Much of the quantum machine learning literature build on algorithms which speed up linear algebra\cite{biamonte2017quantum}.
One of the potentials of quantum machine learning rests upon QRAM -- quantum random access memory. For classical data processing, by using QRAM we may be able to represent a large amount of classical data, N-bits, using $\log$N qubits. However, the bottleneck may be in the encoding/decoding of the QRAM, which may eliminate potential advantages. 
Quantum machine learning may find applications in a more natural setting where both the input an output are quantum states, for example, to control a quantum system, or in learning probability distributions where entanglement plays an important role.

\subsection{Quantum semidefinite programming}

Semidefinite programming is the task of optimizing a linear function, given some matrix inequality constraints. Classically, the problem can be solved in an amount of time that 
is polynomial in the size of the matrix, and the number of constraints.

A quantum algorithm has shown to find an approximate solution to the problem with an exponential speed up \cite{brandao2017quantum,brandao2017exponential}. In the algorithm, the initial state is a thermal state that is a function of the input matrices for the semidefinite program. The success of the implementation depends on whether the particular thermal state can be efficiently prepared. 
The output is a quantum state $\rho$ which approximates the optimal matrix. The quantum state can be measured to extract information on the matrix. 


The crucial feature in the quantum algorithm is the preparation of a thermal state at a nonzero temperature. This suggests that the algoithm may be intrinsically robust against thermal noise. It is therefore entirely possible that a quantum solver for semidefinite programs might be achievable with NISQ technology.


\subsection{Quantum dynamics}
As it has been stressed previously, quantum computers are very well suited for studying highly entangled systems of many particles. It is the natural platform to simulate entangled states, which is where quantum computers appear to have a clear advantage over classical ones.
 
With a universal quantum computer, we anticipate that quantum chemistry will be facilitated by quantum computing. This can be used to design farmaceuticals, as well as new catalysts which can improve the efficiency of nitrogen fixation or carbon capture. We may be able to find new materials that can lead to more efficient electricity transmission. However, these promises may not be fulfilled with NISQ, because algorithms to accurately simulate large molecules and materials may not succeed without quantum error correction.

However, we know that classical computers are particularly inefficient at simulating quantum dynamics, i.e how highly entangled quantum states will evolve time. Here quantum computers have a particularly obvious advantage, and one example would be quantum chaos. In these systems entanglement spreads very rapidly. Insights can be gained using noisy devices with orders of 100's of qubits.

% We have barely had a glimpse of the promises of quantum technology. 
