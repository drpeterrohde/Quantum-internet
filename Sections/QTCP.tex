%
% Quantum Transmission Control Protocol (QTCP)
%

\section{Quantum Transmission Control Protocol (QTCP)} \index{Quantum Transmission Control Protocol (QTCP)}\label{sec:QTCP}

In the classical world, TCP is employed for data transmission and routing. Next we present a simple toy model for a proposed quantum equivalent -- the Quantum Transmission Control Protocol (QTCP). The stack is described in detail in Sec.~\ref{sec:prot_stack}.

We emphasise that this toy model is not intended to be a proposal suited to immediate implementation, solving all the problems of quantum communication in the most effective way. Rather, we simply aim to construct a sketch of the data structures and algorithms that might form a basis for future, more well-considered real-world implementations. Alternately, it is plausible that a QTCP protocol may never reach the light of day at all, instead being made redundant by networks built entirely on entanglement distribution, discussed in detail in Secs.~\ref{sec:rep_net} \& \ref{sec:ent_ultimate}. The answer to this question is difficult to foresee, largely depending on the future requirements of quantum communication protocols.

The goal of QTCP is to abstract away the low-level physical operation of a quantum network to create a virtual interface between Alice and Bob, allowing direct access to data as if it were held locally, in much the same way that high-level services like classical FTP facilitate interaction with remote data as though it were a local asset, blind to the intermediate networking.

The design goals of our elementary toy model are simply to capture the quintessential feature requirements of real-world protocols, and a sketch for their implementation. The designs we present should not be interpreted as a final proposal, but merely as laying a foundation of ideas to build upon.

We consider the scenario where Alice (or a set of Alices) is in possession of some quantum state, which she wishes to communicate to Bob (Bobs), with the aim of optimising some arbitrary cost measure. Bob is no guru and doesn't want to concern himself with how the state was communicated from Alice to himself -- his only concern is that he receives it and that it satisfies quality constraints he and Alice have agreed upon.

QTCP is the joint software/hardware stack that facilitates these objectives. QTCP begins by logically separating different levels of network functionality into distinct layers of abstraction. This includes primarily:
\begin{itemize}
	\item Encapsulation of data into packets of quantum information (Secs.~\ref{sec:data_message_layer} \& \ref{sec:packet_layer}).
	\item Cost vector analysis (Sec.~\ref{sec:costs}).
	\item Routing decisions (Secs.~\ref{sec:intro_strat} \& \ref{sec:strategies}).
	\item Reconstruction of communicated quantum states upon receipt (Sec.~\ref{sec:reconstruction_layer}).
	\item Enforcement of quality of service requirements and error correction (Sec.~\ref{sec:reconstruction_layer}).
	\item Providing a high-level virtual interface between end-users, which abstracts away low-level operations (Sec.~\ref{sec:services_apps}).
\end{itemize}

All the while, Alice and Bob, as end-users, ought to be as blind as possible to the lower-level layers, instead only directly interfacing with the highest layer of abstraction, that which provides the virtual interface between end-users.